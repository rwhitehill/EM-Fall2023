\def\duedate{\today}
\def\HWnum{10}
\input{../preamble.tex}

\begin{document}

\prob{1}{

Two concentric conducting spheres of inner and outer radii $a$ and $b$, respectively, carry charges $\pm Q$.
The empty space between the spheres is half-filled by a hemispherical shell of dielectric (of dielectric constant $\epsilon/\epsilon_0$), as shown in the figure.

(a) Find the electric field everywhere between the spheres.

(b) Calculate the surface-charge distribution on the inner sphere.

(c) Calculate the polarization-charge density induced on the surface of the dielectric at $r = a$.

}

\sol{}


\prob{2}{

A right-circular solendoid of finite length $L$ and radius $a$ has $N$ turns per unit length and carries a current $I$.
Show that the magnetic induction on the cylinder axis on the cylinder axis in the limit $NL \rightarrow \infty$ is
\begin{eqnarray}
    B_{z} = \frac{\mu_0 N I}{2} ( \cos{\theta_1} + \cos{\theta_2} )
,\end{eqnarray}
where $\theta_{1,2}$ are defined in the figure.

}

\sol{

Let us define the left end of the cylinder to be at position $z = 0$ and the right to be at $z = L$.
At position $z$, the angles $\theta_1$ and $\theta_2$ are defined as
\begin{eqnarray}
    \cos{\theta_1} = \frac{z}{\sqrt{z^2 + a^2}}, \quad \cos{\theta_2} = \frac{L-z}{\sqrt{(L-z)^2 + a^2}}
.\end{eqnarray}

Amp\`{e}re's law in integral form (for magetostatics) is
\begin{eqnarray}
    \int \va*{B} \cdot \dd{\va*{l}} = \mu_0 I_{\rm enc}
.\end{eqnarray}
If we place a rectangular loop that straddles only the upper portion of the solenoid, then we have
\begin{eqnarray}
    d
.\end{eqnarray}



}


\prob{3}{

A cylindrical conductor of radius $a$ has a hole of radius $b$ bored parallel to, and centered a distance $d$ from, the cylinder axis ($d + b < a$).
The current density is uniform throughout the remaining metal of the cylinder and is parallel to the axis.
Use Amp\`{e}re's law and principle of linear superposition to find the magnitude and the direction of the magnetic-flux density in the hole.

}


\sol{}



\end{document}
