\def\duedate{\today}
\def\HWnum{10}
\documentclass[10pt,a4paper]{book}

% custom section formatting
\usepackage{titlesec}
\titleformat{\chapter}[display]
{\normalfont\Large\filcenter\sffamily}
{\titlerule[1pt]%
\vspace{1pt}%
\titlerule
\vspace{1pc}%
\LARGE\MakeUppercase{\chaptertitlename} \thechapter}
{1pc}
{\titlerule
\vspace{1pc}%
\Huge}

% appendix handling
\usepackage[toc,page]{appendix}
    
% encoding for file and font
\usepackage[utf8]{inputenc}
\usepackage[T1]{fontenc}

% math formatting/tools
\usepackage{amsmath}
\usepackage{amssymb}
\usepackage{mathtools}
\usepackage[arrowdel]{physics}

% unit formatting
\usepackage{siunitx}
\AtBeginDocument{\RenewCommandCopy\qty\SI}

% figure formatting/tools
\usepackage{graphicx}
\usepackage{float}
\usepackage{subcaption}
\usepackage{multirow}
\usepackage{import}
\usepackage{pdfpages}
\usepackage{transparent}
\usepackage{currfile}

\NewDocumentCommand\incfig{O{1} m}{
    \def\svgwidth{#1\textwidth}
    \import{./Figures/\currfiledir}{#2.pdf_tex}
}

\newcommand{\bef}{\begin{figure}[h!tb]\centering}
\newcommand{\eef}{\end{figure}}

\newcommand{\bet}{\begin{table}[h!tb]\centering}
\newcommand{\eet}{\end{table}}

% hyperlink references 
\usepackage{hyperref}
\hypersetup{
    colorlinks=true,
    linkcolor=blue,
    filecolor=magenta,
    urlcolor=cyan,
    pdftitle={Physics 1 Notes},
    pdfauthor={Richard Whitehill},
    pdfpagemode=FullScreen
}
\urlstyle{same}

\newcommand{\eref}[1]{Eq.~(\ref{eq:#1})}
\newcommand{\erefs}[2]{Eqs.~(\ref{eq:#1})--(\ref{eq:#2})}

\newcommand{\fref}[1]{Fig.~(\ref{fig:#1})}
\newcommand{\frefs}[2]{Fig.~(\ref{fig:#1})--(\ref{fig:#2})}

\newcommand{\aref}[1]{Appendix~(\ref{app:#1})}
\newcommand{\sref}[1]{Section~(\ref{sec:#1})}
\newcommand{\srefs}[2]{Sections~(\ref{sec:#1})-(\ref{sec:#2})}

\newcommand{\tref}[1]{Table~(\ref{tab:#1})}
\newcommand{\trefs}[2]{Table~(\ref{tab:#1})--(\ref{tab:#2})}

% tcolorbox formatting/definitions
\usepackage[most]{tcolorbox}
\usepackage{xcolor}
\usepackage{xifthen}
\usepackage{parskip}

\definecolor{peach}{rgb}{1.0,0.8,0.64}

\DeclareTColorBox[auto counter, number within=chapter]{defbox}{O{}}{
    enhanced,
    boxrule=0pt,
    frame hidden,
    borderline west={4pt}{0pt}{green!50!black},
    colback=green!5,
    before upper=\textbf{Definition \thetcbcounter \ifthenelse{\isempty{#1}}{}{: #1} \\ },
    sharp corners
}

\newcommand*{\eqbox}{\tcboxmath[
    enhanced,
    colback=black!10!white,
    colframe=black,
    sharp corners,
    size=fbox,
    boxsep=8pt,
    boxrule=1pt
]}

\newtcolorbox[auto counter, number within=chapter]{exbox}{
    parbox=false,
    breakable,
    enhanced,
    sharp corners,
    boxrule=1pt,
    colback=white,
    colframe=black,
    before upper= \textbf{Example \thetcbcounter:}\,,
    before lower= \textbf{Solution:}\,,
    segmentation hidden
}

\newtcolorbox{resbox}{
    enhanced,
    colback=black!10!white,
    colframe=black,
    boxrule=1pt,
    boxsep=0pt,
    top=2pt,
    ams nodisplayskip,
    sharp corners
}


\begin{document}

\prob{1}{

Two concentric conducting spheres of inner and outer radii $a$ and $b$, respectively, carry charges $\pm Q$.
The empty space between the spheres is half-filled by a hemispherical shell of dielectric (of dielectric constant $\epsilon/\epsilon_0$), as shown in the figure.

(a) Find the electric field everywhere between the spheres.

(b) Calculate the surface-charge distribution on the inner sphere.

(c) Calculate the polarization-charge density induced on the surface of the dielectric at $r = a$.

}

\sol{}


\prob{2}{

A right-circular solendoid of finite length $L$ and radius $a$ has $N$ turns per unit length and carries a current $I$.
Show that the magnetic induction on the cylinder axis on the cylinder axis in the limit $NL \rightarrow \infty$ is
\begin{eqnarray}
    B_{z} = \frac{\mu_0 N I}{2} ( \cos{\theta_1} + \cos{\theta_2} )
,\end{eqnarray}
where $\theta_{1,2}$ are defined in the figure.

}

\sol{

Let us define the left end of the cylinder to be at position $z = 0$ and the right to be at $z = L$.
At position $z$, the angles $\theta_1$ and $\theta_2$ are defined as
\begin{eqnarray}
    \cos{\theta_1} = \frac{z}{\sqrt{z^2 + a^2}}, \quad \cos{\theta_2} = \frac{L-z}{\sqrt{(L-z)^2 + a^2}}
.\end{eqnarray}

Amp\`{e}re's law in integral form (for magetostatics) is
\begin{eqnarray}
    \int \va*{B} \cdot \dd{\va*{l}} = \mu_0 I_{\rm enc}
.\end{eqnarray}
If we place a rectangular loop that straddles only the upper portion of the solenoid, then we have
\begin{eqnarray}
    d
.\end{eqnarray}



}


\prob{3}{

A cylindrical conductor of radius $a$ has a hole of radius $b$ bored parallel to, and centered a distance $d$ from, the cylinder axis ($d + b < a$).
The current density is uniform throughout the remaining metal of the cylinder and is parallel to the axis.
Use Amp\`{e}re's law and principle of linear superposition to find the magnitude and the direction of the magnetic-flux density in the hole.

}


\sol{}



\end{document}
