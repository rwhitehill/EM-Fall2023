\def\duedate{\today}
\def\HWnum{10}
\documentclass[10pt,a4paper]{book}

% custom section formatting
\usepackage{titlesec}
\titleformat{\chapter}[display]
{\normalfont\Large\filcenter\sffamily}
{\titlerule[1pt]%
\vspace{1pt}%
\titlerule
\vspace{1pc}%
\LARGE\MakeUppercase{\chaptertitlename} \thechapter}
{1pc}
{\titlerule
\vspace{1pc}%
\Huge}

% appendix handling
\usepackage[toc,page]{appendix}
    
% encoding for file and font
\usepackage[utf8]{inputenc}
\usepackage[T1]{fontenc}

% math formatting/tools
\usepackage{amsmath}
\usepackage{amssymb}
\usepackage{mathtools}
\usepackage[arrowdel]{physics}

% unit formatting
\usepackage{siunitx}
\AtBeginDocument{\RenewCommandCopy\qty\SI}

% figure formatting/tools
\usepackage{graphicx}
\usepackage{float}
\usepackage{subcaption}
\usepackage{multirow}
\usepackage{import}
\usepackage{pdfpages}
\usepackage{transparent}
\usepackage{currfile}

\NewDocumentCommand\incfig{O{1} m}{
    \def\svgwidth{#1\textwidth}
    \import{./Figures/\currfiledir}{#2.pdf_tex}
}

\newcommand{\bef}{\begin{figure}[h!tb]\centering}
\newcommand{\eef}{\end{figure}}

\newcommand{\bet}{\begin{table}[h!tb]\centering}
\newcommand{\eet}{\end{table}}

% hyperlink references 
\usepackage{hyperref}
\hypersetup{
    colorlinks=true,
    linkcolor=blue,
    filecolor=magenta,
    urlcolor=cyan,
    pdftitle={Physics 1 Notes},
    pdfauthor={Richard Whitehill},
    pdfpagemode=FullScreen
}
\urlstyle{same}

\newcommand{\eref}[1]{Eq.~(\ref{eq:#1})}
\newcommand{\erefs}[2]{Eqs.~(\ref{eq:#1})--(\ref{eq:#2})}

\newcommand{\fref}[1]{Fig.~(\ref{fig:#1})}
\newcommand{\frefs}[2]{Fig.~(\ref{fig:#1})--(\ref{fig:#2})}

\newcommand{\aref}[1]{Appendix~(\ref{app:#1})}
\newcommand{\sref}[1]{Section~(\ref{sec:#1})}
\newcommand{\srefs}[2]{Sections~(\ref{sec:#1})-(\ref{sec:#2})}

\newcommand{\tref}[1]{Table~(\ref{tab:#1})}
\newcommand{\trefs}[2]{Table~(\ref{tab:#1})--(\ref{tab:#2})}

% tcolorbox formatting/definitions
\usepackage[most]{tcolorbox}
\usepackage{xcolor}
\usepackage{xifthen}
\usepackage{parskip}

\definecolor{peach}{rgb}{1.0,0.8,0.64}

\DeclareTColorBox[auto counter, number within=chapter]{defbox}{O{}}{
    enhanced,
    boxrule=0pt,
    frame hidden,
    borderline west={4pt}{0pt}{green!50!black},
    colback=green!5,
    before upper=\textbf{Definition \thetcbcounter \ifthenelse{\isempty{#1}}{}{: #1} \\ },
    sharp corners
}

\newcommand*{\eqbox}{\tcboxmath[
    enhanced,
    colback=black!10!white,
    colframe=black,
    sharp corners,
    size=fbox,
    boxsep=8pt,
    boxrule=1pt
]}

\newtcolorbox[auto counter, number within=chapter]{exbox}{
    parbox=false,
    breakable,
    enhanced,
    sharp corners,
    boxrule=1pt,
    colback=white,
    colframe=black,
    before upper= \textbf{Example \thetcbcounter:}\,,
    before lower= \textbf{Solution:}\,,
    segmentation hidden
}

\newtcolorbox{resbox}{
    enhanced,
    colback=black!10!white,
    colframe=black,
    boxrule=1pt,
    boxsep=0pt,
    top=2pt,
    ams nodisplayskip,
    sharp corners
}


\begin{document}

\prob{1}{

Two concentric conducting spheres of inner and outer radii $a$ and $b$, respectively, carry charges $\pm Q$.
The empty space between the spheres is half-filled by a hemispherical shell of dielectric (of dielectric constant $\epsilon/\epsilon_0$), as shown in the figure.

(a) Find the electric field everywhere between the spheres.

(b) Calculate the surface-charge distribution on the inner sphere.

(c) Calculate the polarization-charge density induced on the surface of the dielectric at $r = a$.

}

\sol{

(a) In the presence of dielectric media, we have a form of Gauss's law in terms of the electric displacement and free charge:
\begin{eqnarray}
    \int_{S} \epsilon(\va*{x}) \va*{E} \vdot \dd{\va*{S}} = Q_{f}
.\end{eqnarray}
Note that the electric field is purely radial and constant inside the volume along surfaces of constant $r$\footnote{This is justified between a few reasons: (1) The electric field must be radial in the half-regions; (2) the tangential component of the electric field at the interface, which is radial, must be constant; (3) the difference across the interface of the normal component of the electric displaycement is equal to the free charge at the interface, of which there is none.}, but the integrand is not constant, so we split the integrand between the regions that are filled and not filled, respectively, yielding (integrating over a spherical Gaussian surface)
\begin{eqnarray}
    2 \pi r^2 (\epsilon + \epsilon_0) E_{r} = Q \Rightarrow \eqbox{ \va*{E} = \frac{Q}{2\pi(\epsilon + \epsilon_0) r^2} \vu*{r} }
.\end{eqnarray}

(b) The charge density on the inner sphere is given by the boundary condition
\begin{eqnarray}
    \eqbox{ \sigma_f = \va*{D} \cdot \vu*{r} = \frac{\epsilon(\va*{x})}{\epsilon + \epsilon_0} \frac{Q}{2 \pi a^2} = \frac{Q}{2 \pi a^2} \begin{cases}
        (1+\epsilon/\epsilon_0)^{-1} & {\rm left} \\
        (1 + \epsilon_0/\epsilon)^{-1} & {\rm right}
    ,\end{cases}
}
\end{eqnarray}
Note that this tells us that the surface next to vacuum has more free surface charge, which must come from the region against the dielectric.
Essentially, some charge migrates from the right region to the left because of the bound surface charge that is created as a result of the polarization of the dielectric.

\newpage 

(c) The bound surface charge at $r = a$ is given by
\begin{eqnarray}
\eqbox{
\begin{aligned} 
    \sigma_b &= -\va*{P} \cdot \vu*{r} = -(\epsilon(\va*{x}) - \epsilon_0) \va*{E} \cdot \vu*{r} = - \frac{\epsilon(\va*{x}) - \epsilon_0}{\epsilon + \epsilon_0} \frac{Q}{2 \pi a^2} \\
             &= \begin{cases}
        0 & {\rm left} \\
        -[(\epsilon - \epsilon_0)/(\epsilon+\epsilon_0)] (Q / 2 \pi a^2) & {\rm right}
    .\end{cases}
\end{aligned}
}
\end{eqnarray}

}


\prob{2}{

A right-circular solendoid of finite length $L$ and radius $a$ has $N$ turns per unit length and carries a current $I$.
Show that the magnetic induction on the cylinder axis on the cylinder axis in the limit $NL \rightarrow \infty$ is
\begin{eqnarray}
    B_{z} = \frac{\mu_0 N I}{2} ( \cos{\theta_1} + \cos{\theta_2} )
,\end{eqnarray}
where $\theta_{1,2}$ are defined in the figure.

}

\sol{

Notice that we can essentially create a solenoid by stacking circular loops in succession which are uniformly spaced.
Thus, we can begin by determining the magnetic field produced by one of these circular loops.
Let the loop have current $I$ and radius $a$.
The Biot-Savart law then tells us that
\begin{eqnarray}
    \va*{B} = \frac{\mu_0}{4 \pi} \int \dd[3]{\va*{x}'} \frac{\va*{J}(\va*{x}') \cross (\va*{x} - \va*{x}')}{|\va*{x} - \va*{x}'|^3} 
.\end{eqnarray}
Since the current is constrained to the loop, we have
\begin{eqnarray}
\begin{aligned} 
    \va*{B} &= \frac{\mu_0}{4\pi} \frac{I a}{[a^2 + z^2]^{3/2}} \int_{0}^{2\pi} \dd{\phi} \vu*{\phi} \cross (z \vu*{z} - a \vu*{s}) = \frac{\mu_0}{4 \pi} \frac{I a}{[a^2 + z^2]^{3/2}} \int_{0}^{2\pi} \dd{\phi} ( z \vu*{s} + a \vu*{z} ) \\
            &= \frac{\mu_0 I}{2} \frac{a^2}{[a^2 + z^2]^{3/2}} \vu*{z}
.\end{aligned}
\end{eqnarray}

Now, to create the solenoid, we can now stack the current loops very densely.
Thus, the total magnetic field is the integral of the contributions from each individual loop.
Let us define the left end of the solendoid to be located at $z = 0$ and the right end to be at $z = L$.
The magnetic field is then
\begin{eqnarray}
\eqbox{
\begin{aligned}
    \va*{B} &= \vu*{z} \frac{\mu_0 N I a^2}{2} \int_{0}^{L} \frac{1}{[a^2 + (z - z')^2]^{3/2}} \dd{z'} = \frac{\mu_0 N I }{2} \Bigg[ \frac{z}{\sqrt{z^2 + a^2}} + \frac{L-z}{\sqrt{(z-L)^2 + a^2}} \Bigg] \vu*{z} \\
            &= \frac{\mu_0 N I}{2} [ \cos{\theta_1} + \cos{\theta_2} ] \vu*{z}
.\end{aligned}
}
\end{eqnarray}
Note that the factor of $N$ appears since the total current that pierces a small area (e.g. a circle of radius $\dd{z}$) perpendicular to the surface of the cylinder is $NI$.

}


\prob{3}{

A cylindrical conductor of radius $a$ has a hole of radius $b$ bored parallel to, and centered a distance $d$ from, the cylinder axis ($d + b < a$).
The current density is uniform throughout the remaining metal of the cylinder and is parallel to the axis.
Use Amp\`{e}re's law and principle of linear superposition to find the magnitude and the direction of the magnetic-flux density in the hole.

}


\sol{

Observe that we can mathematically create this hole by superimposing the large cylinder with radius $a$ carrying a current density $\va*{J} = J \vu*{z}$, where $\vu*{z}$ is the unit vector along the axis of the cylinder, and a smaller cylinder with radius $b$ a distance $d$ from the center of the large cylinder with current density $-\va*{J}$ (i.e. the same magnitude but opposite direction relative to the current of the large cylinder).
Ampere's law states that
\begin{eqnarray}
    \oint_{C} \va*{B} \vdot \dd{\va*{l}} = \mu_0 \int_{S} \va*{J} \vdot \dd{\va*{S}}
.\end{eqnarray}
We want to find $B(\va*{x})$, where $\va*{x}$ is a point inside the small cylinder.
For the large cylinder, let us select a circular Amperian loop of radius $|\va*{x}| = r$ centered on the large cylinder's axis.
We then have
\begin{eqnarray}
    2 \pi r B_{\phi} = \mu_0 \pi r^2 J \Rightarrow \va*{B}_1 = \frac{\mu_0 J r}{2} \vu*{\phi} = \frac{\mu_0 J}{2} [ -y \vu*{x} + x \vu*{y} ]
.\end{eqnarray}
Doing the same thing with the smaller cylinder, we select a circle of radius $r'$ centered on the small cylinder's axis:
\begin{eqnarray}
    \va*{B}_2 = -\frac{\mu_0 J r'}{2} \vu*{\phi}' = -\frac{\mu_0 J}{2} [ -y' \vu*{x} + x' \vu*{y} ] = \frac{\mu_0 J}{2} [ (y - d_{y}) \vu*{x} + (d_{x} - x) \vu*{y} ]
.\end{eqnarray}
We can superimpose these magnetic fields to find
\begin{eqnarray}
\eqbox{
\begin{aligned} 
    \va*{B} &= \va*{B}_{1} + \va*{B}_{2} = \frac{\mu_0 J}{2} [ -d_{y} \vu*{x} + d_{x} \vu*{y} ] \\
            &= \frac{\mu_0 J d}{2} [ -\sin{\phi} \vu*{x} + \cos{\phi} \vu*{y} ] = \frac{\mu_0 J d}{2} \vu*{\phi}
.\end{aligned}
}
\end{eqnarray}
Note that $\vu*{\phi}$ is the polar unit vector at the center of the small circle with the origin of the coordinate system at the center of the large cylinder.
The direction of the unit vector is given by the right-hand-rule.
Observe that our result tells us that the magnetic field is constant inside of the hole.

}



\end{document}
