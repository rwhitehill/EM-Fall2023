\chapter{Formal Solution of Boundary-Value Problems Using Green's Functions}

In this section we outline a more formally and rigorous way of writing down solutions to Poisson's and Laplace's equations.
Observe the following general fact:
\begin{align}
    \laplacian \frac{1}{|\va*{x} - \va*{x}_{0}|} = -4 \pi \delta(\va*{x} - \va*{x}_{0})
.\end{align}

We begin with Green's theorem
\begin{align}
    \int_{V} &\dd[3]{\va*{x}'} [ \psi_1(\va*{x}') \nabla'^2 \psi_2(\va*{x}') - \psi_2(\va*{x}') \nabla'^2 \psi_1(\va*{x}') ] \nonumber \\
             &= \int_{S} \dd{S'} \Big[ \psi_1(\va*{x}') \pdv{\psi_2(\va*{x}')}{n'} - \psi_2(\va*{x}') \pdv{\psi_1(\va*{x}')}{n'} \Big]
.\end{align}

If we let $\psi_1(\va*{x}') = |\va*{x} - \va*{x}'|^{-1}$ and take $\psi_2(\va*{x}') = \Phi(\va*{x}')$, then Green's theorem reduces to 
\begin{align}
    \int_{V} &\dd[3]{\va*{x}'} \Bigg[ \frac{1}{|\va*{x} - \va*{x}'|} \nabla'^2 \Phi(\va*{x}') - \Phi(\va*{x}') \nabla'^2 \frac{1}{|\va*{x} - \va*{x}'|} \Bigg] \nonumber \\
             &= \int_{S} \dd{S'} \Bigg[ \frac{1}{|\va*{x} - \va*{x}'|} \pdv{\Phi(\va*{x}')}{n'} - \Phi(\va*{x}') \pdv{n'} \frac{1}{|\va*{x} - \va*{x}'|} \Bigg]
.\end{align}
Simplifying and rearranging gives
\begin{align}
    \Phi(\va*{x}) &= \frac{1}{4 \pi \epsilon_0} \int_{V} \frac{\rho(\va*{x}')}{|\va*{x} - \va*{x}'|} \dd[3]{\va*{x}'} \nonumber \\
                  &+ \frac{1}{4 \pi} \int_{S} \Bigg[ \frac{1}{|\va*{x} - \va*{x}'|} \pdv{\Phi(\va*{x}')}{n'} - \Phi(\va*{x}') \pdv{n'} \frac{1}{|\va*{x} - \va*{x}'|} \Bigg] \dd{S'}
.\end{align}
At this point, we can solve for the potential due to a source charge distribution and by specifying the boundary conditions for $\Phi$.
The function $G(\va*{x},\va*{x}') = |\va*{x} - \va*{x}'|^{-1}$ is called the Green's function for this problem.
Note that the Green's function is not unique.
It simply satisfies
\begin{align}
    \nabla'^2 G(\va*{x},\va*{x}') = -4 \pi \delta(\va*{x} - \va*{x}')
,\end{align}
so we can always shift our Green's function as $G \rightarrow G + F$, where $F(\va*{x},\va*{x}')$ is a solution of Laplace's equation.
Thus, we can generally write
\begin{align}
    \label{eq:phi-G}
    \Phi(\va*{x}) &= \frac{1}{4 \pi \epsilon_0} \int_{V} G(\va*{x},\va*{x}') \rho(\va*{x}') \dd[3]{\va*{x}'} \nonumber \\
                  &+ \frac{1}{4 \pi} \int_{S} \Bigg[ G(\va*{x},\va*{x}') \pdv{\Phi(\va*{x}')}{n'} - \Phi(\va*{x}') \pdv{G(\va*{x},\va*{x}')}{n'} \Bigg] \dd{S'}
.\end{align}
We will exploit this fact to make solving \eref{phi-G} simpler in the context of a specific set of BCs since we cannot simultaneously specify Dirichlet or Neumann BCs.

\section{Diriclet Problem}

If we specify the value of $\Phi$ on the surface $S$, we choose $G(\va*{x},\va*{x}') = G_{D}(\va*{x},\va*{x}')$ such that $G_{D}(\va*{x},\va*{x}') = 0$ if $\va*{x} \in S$.
\eref{phi-G} then reduces to
\begin{align}
    \Phi(\va*{x}) = \frac{1}{4 \pi \epsilon_0} \int_{V} G_{D}(\va*{x},\va*{x}') \rho(\va*{x}') \dd[3]{\va*{x}'} - \frac{1}{4 \pi} \int_{S} \Phi(\va*{x}') \pdv{G_{D}(\va*{x},\va*{x}')}{n'} \dd{S'}
.\end{align}

One useful fact about $G_{D}$ is that it satisfies a reciprocity relation: $G_{D}(\va*{x},\va*{y}) = G_{D}(\va*{y},\va*{x})$.
This can be seen by Green's theorem with the choice $\psi_1(\va*{x}') = G_{D}(\va*{x},\va*{x}')$ and $\psi_2(\va*{x}') = G_{D}(\va*{x}',\va*{y})$:
\begin{align}
    \int_{V} &\dd[3]{\va*{x}'} \big[ G_{D}(\va*{x},\va*{x}') \nabla'^2 G_{D}(\va*{x}',\va*{y}) - G_{D}(\va*{x}',\va*{y}) \nabla'^2 G_{D}(\va*{x},\va*{x}') \big] \nonumber \\
             &= \int_{S} \dd{S'} \vu*{n}' \cdot \Bigg[ G_{D}(\va*{x},\va*{x}') \grad' G_{D}(\va*{x}',\va*{y}) - G_{D}(\va*{x}',\va*{y}) \grad' G_{D}(\va*{x},\va*{x}') \Bigg]
.\end{align}
Thus, using the definition of the Diriclet Green's function
\begin{eqnarray}
    -4 \pi \big[ G_{D}(\va*{x},\va*{y}) - G_{D}(\va*{y},\va*{x}) \big] = 0 \Rightarrow G_{D}(\va*{x},\va*{y}) = G_{D}(\va*{y},\va*{x})
.\end{eqnarray}


\section{Neumann Problem}

Suppose we instead specify the value of $\partial \Phi/ \partial n$ on the surface.
In this case, it is tempting to make a similar argument as in the previous section and eliminate the second term in the surface integration such that we only have the first term.
But this is not generally possible.
Observe that $G_{N}$ must satisfy
\begin{align}
    \int_{S} \dd{S'} \pdv{G_{N}(\va*{x},\va*{x}')}{n'} = \int_{V} \dd[3]{\va*{x}'} \nabla'^2 G_{N}(\va*{x},\va*{x}') = - 4 \pi
,\end{align}
and hence, it is not possible to assert that $\partial G_{N} / \partial n \equiv 0$.

Another choice that makes the problem solvable is by choosing $G_{N}$ such that
\begin{align}
    \pdv{G_{N}(\va*{x},\va*{x}')}{n'} = -\frac{4 \pi}{S} ~{\rm for}~\va*{x}' \in S
,\end{align}
where $S$ (in the denominator) is the surface area of $S$.
In this case, the solution becomes
\begin{align}
    \Phi(\va*{x}) &= \frac{1}{4 \pi \epsilon_0} \int_{V} \dd[3]{\va*{x}'} G_{N}(\va*{x},\va*{x}') \rho(\va*{x}') + \frac{1}{4 \pi} \int_{S} \dd{S'} G_{N}(\va*{x},\va*{x}') \pdv{\Phi(\va*{x}')}{n'} \\
                  &+ \frac{1}{S} \int_{S} \dd{S'} \Phi(\va*{x}')
.\end{align}
Notice that the last term is just the average value of $\Phi$ over the surface $S$.

\section{Methods of Finding Green's Functions'}





