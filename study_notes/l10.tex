\chapter{Formal Solution of Boundary-Value Problems Using Green's Functions}

In this section we outline a more formally and rigorous way of writing down solutions to Poisson's and Laplace's equations.
Observe the following general fact:
\begin{align}
    \laplacian \frac{1}{|\va*{x} - \va*{x}_{0}|} = -4 \pi \delta(\va*{x} - \va*{x}_{0})
.\end{align}

We begin with Green's theorem
\begin{align}
    \int_{V} &\dd[3]{\va*{x}'} [ \psi_1(\va*{x}') \nabla'^2 \psi_2(\va*{x}') - \psi_2(\va*{x}') \nabla'^2 \psi_1(\va*{x}') ] \nonumber \\
             &= \int_{S} \dd{S'} \Big[ \psi_1(\va*{x}') \pdv{\psi_2(\va*{x}')}{n'} - \psi_2(\va*{x}') \pdv{\psi_1(\va*{x}')}{n'} \Big]
.\end{align}

If we let $\psi_1(\va*{x}') = |\va*{x} - \va*{x}'|^{-1}$ and take $\psi_2(\va*{x}') = \Phi(\va*{x}')$, then Green's theorem reduces to 
\begin{align}
    \int_{V} &\dd[3]{\va*{x}'} \Bigg[ \frac{1}{|\va*{x} - \va*{x}'|} \nabla'^2 \Phi(\va*{x}') - \Phi(\va*{x}') \nabla'^2 \frac{1}{|\va*{x} - \va*{x}'|} \Bigg] \nonumber \\
             &= \int_{S} \dd{S'} \Bigg[ \frac{1}{|\va*{x} - \va*{x}'|} \pdv{\Phi(\va*{x}')}{n'} - \Phi(\va*{x}') \pdv{n'} \frac{1}{|\va*{x} - \va*{x}'|} \Bigg]
.\end{align}
Simplifying and rearranging gives
\begin{align}
    \Phi(\va*{x}) &= \frac{1}{4 \pi \epsilon_0} \int_{V} \frac{\rho(\va*{x}')}{|\va*{x} - \va*{x}'|} \dd[3]{\va*{x}'} \nonumber \\
                  &+ \frac{1}{4 \pi} \int_{S} \Bigg[ \frac{1}{|\va*{x} - \va*{x}'|} \pdv{\Phi(\va*{x}')}{n'} - \Phi(\va*{x}') \pdv{n'} \frac{1}{|\va*{x} - \va*{x}'|} \Bigg] \dd{S'}
.\end{align}
At this point, we can solve for the potential due to a source charge distribution and by specifying the boundary conditions for $\Phi$.
The function $G(\va*{x},\va*{x}') = |\va*{x} - \va*{x}'|^{-1}$ is called the Green's function for this problem.
Note that the Green's function is not unique.
It simply satisfies
\begin{align}
    \nabla'^2 G(\va*{x},\va*{x}') = -4 \pi \delta(\va*{x} - \va*{x}')
,\end{align}
so we can always shift our Green's function as $G \rightarrow G + F$, where $F(\va*{x},\va*{x}')$ is a solution of Laplace's equation.
Thus, we can generally write
\begin{align}
    \label{eq:phi-G}
    \Phi(\va*{x}) &= \frac{1}{4 \pi \epsilon_0} \int_{V} G(\va*{x},\va*{x}') \rho(\va*{x}') \dd[3]{\va*{x}'} \nonumber \\
                  &+ \frac{1}{4 \pi} \int_{S} \Bigg[ G(\va*{x},\va*{x}') \pdv{\Phi(\va*{x}')}{n'} - \Phi(\va*{x}') \pdv{G(\va*{x},\va*{x}')}{n'} \Bigg] \dd{S'}
.\end{align}
We will exploit this fact to make solving \eref{phi-G} simpler in the context of a specific set of BCs since we cannot simultaneously specify Dirichlet or Neumann BCs.

\section{Diriclet Problem}

If we specify the value of $\Phi$ on the surface $S$, we choose $G(\va*{x},\va*{x}') = G_{D}(\va*{x},\va*{x}')$ such that $G_{D}(\va*{x},\va*{x}') = 0$ if $\va*{x} \in S$.
\eref{phi-G} then reduces to
\begin{align}
    \Phi(\va*{x}) = \frac{1}{4 \pi \epsilon_0} \int_{V} G_{D}(\va*{x},\va*{x}') \rho(\va*{x}') \dd[3]{\va*{x}'} - \frac{1}{4 \pi} \int_{S} \Phi(\va*{x}') \pdv{G_{D}(\va*{x},\va*{x}')}{n'} \dd{S'}
.\end{align}

One useful fact about $G_{D}$ is that it satisfies a reciprocity relation: $G_{D}(\va*{x},\va*{y}) = G_{D}(\va*{y},\va*{x})$.
This can be seen by Green's theorem with the choice $\psi_1(\va*{x}') = G_{D}(\va*{x},\va*{x}')$ and $\psi_2(\va*{x}') = G_{D}(\va*{x}',\va*{y})$:
\begin{align}
    \int_{V} &\dd[3]{\va*{x}'} \big[ G_{D}(\va*{x},\va*{x}') \nabla'^2 G_{D}(\va*{x}',\va*{y}) - G_{D}(\va*{x}',\va*{y}) \nabla'^2 G_{D}(\va*{x},\va*{x}') \big] \nonumber \\
             &= \int_{S} \dd{S'} \vu*{n}' \cdot \Bigg[ G_{D}(\va*{x},\va*{x}') \grad' G_{D}(\va*{x}',\va*{y}) - G_{D}(\va*{x}',\va*{y}) \grad' G_{D}(\va*{x},\va*{x}') \Bigg]
.\end{align}
Thus, using the definition of the Diriclet Green's function
\begin{eqnarray}
    -4 \pi \big[ G_{D}(\va*{x},\va*{y}) - G_{D}(\va*{y},\va*{x}) \big] = 0 \Rightarrow G_{D}(\va*{x},\va*{y}) = G_{D}(\va*{y},\va*{x})
.\end{eqnarray}


\section{Neumann Problem}

Suppose we instead specify the value of $\partial \Phi/ \partial n$ on the surface.
In this case, it is tempting to make a similar argument as in the previous section and eliminate the second term in the surface integration such that we only have the first term.
But this is not generally possible.
Observe that $G_{N}$ must satisfy
\begin{align}
    \int_{S} \dd{S'} \pdv{G_{N}(\va*{x},\va*{x}')}{n'} = \int_{V} \dd[3]{\va*{x}'} \nabla'^2 G_{N}(\va*{x},\va*{x}') = - 4 \pi
,\end{align}
and hence, it is not possible to assert that $\partial G_{N} / \partial n \equiv 0$.

Another choice that makes the problem solvable is by choosing $G_{N}$ such that
\begin{align}
    \pdv{G_{N}(\va*{x},\va*{x}')}{n'} = -\frac{4 \pi}{S} ~{\rm for}~\va*{x}' \in S
,\end{align}
where $S$ (in the denominator) is the surface area of $S$.
In this case, the solution becomes
\begin{align}
    \Phi(\va*{x}) &= \frac{1}{4 \pi \epsilon_0} \int_{V} \dd[3]{\va*{x}'} G_{N}(\va*{x},\va*{x}') \rho(\va*{x}') + \frac{1}{4 \pi} \int_{S} \dd{S'} G_{N}(\va*{x},\va*{x}') \pdv{\Phi(\va*{x}')}{n'} \\
                  &+ \frac{1}{S} \int_{S} \dd{S'} \Phi(\va*{x}')
.\end{align}
Notice that the last term is just the average value of $\Phi$ over the surface $S$.

\section{Methods of Finding Green's Functions}

Consider finding the potential above the $xy$-plane with Dirichlet BCs on the plane.
We want two things
\begin{enumerate}
    \item $G_{D}(\va*{x},\va*{x}') = 0$ if $z' = 0$.
    \item $\nabla'^2 G_{D}(\va*{x},\va*{x}') = - 4 \pi \delta(\va*{x} - \va*{x}')$
\end{enumerate}
Recall that we essentially solved this problem in the section on the method of images:
\begin{eqnarray}
    G_{D}(\va*{x},\va*{x}') = \frac{1}{|\va*{x}' - \va*{x}|} - \frac{1}{|\va*{x}' - \va*{y}|}
,\end{eqnarray}
where $\va*{y}$ is $\va*{x}$ reflected about the $xy$-plane (i.e. the $z$ component differs by a minus sign).
If we suppose further that there are no source charges above the $xy$-plane and that the plane is held at fixed potential $V$ for $x > 0$ and grounded on $x < 0$, then 
\begin{eqnarray}
    \Phi(\va*{x}) = -\frac{V}{4 \pi} \int_{S} \dd{S'} \pdv{G_{D}(\va*{x},\va*{x}')}{n'}
.\end{eqnarray}
The normal derivative in the integrand is
\begin{align}
    \pdv{G_{D}}{n'}\Big|_{z' = 0} = -\pdv{G_{D}}{z'}\Big|_{z' = 0} = -\frac{2z}{[(x'-x)^2 + (y'-y)^2 + z^2]^{3/2}}
,\end{align}
and therefore
\begin{align}
    \Phi(\va*{x}) = \frac{V z}{2 \pi} \int_{0}^{\infty} \int_{-\infty}^{\infty} \dd{x'} \dd{y'} \frac{1}{[(x'-x)^2 + (y'-y)^2 + z^2]^{3/2}}
.\end{align}
If we use the substitutions $u = x' - x$ and $v = y' - y$
\begin{align}
    \label{eq:phi-half-plane}
    \Phi(\va*{x}) = \frac{Vz}{2 \pi} \int_{-x}^{\infty} \int_{-\infty}^{\infty} \dd{u} \dd{v} \frac{1}{[u^2 + v^2 + z^2]^{3/2}}
.\end{align}
Solving the $v$ integral first, let us denote $\xi^2 = u^2 + z^2$, which gives
\begin{eqnarray}
    \Phi(\va*{x}) = \frac{V z}{2 \pi} \int_{-x}^{\infty} \dd{u} \Bigg[ \int_{-\infty}^{\infty} \frac{\dd{v}}{[v^2 + \xi^2]^{3/2}}  \Bigg]
.\end{eqnarray}
If we make the substitution $v = \xi \tan{\theta}$, then
\begin{eqnarray}
    \int_{-\infty}^{\infty} \frac{\dd{v}}{[v^2 + \xi^2]^{3/2}} = \int_{-\pi/2}^{\pi/2} \frac{\xi \sec^2{\theta} \dd{\theta}}{\xi^3 \sec^3{\theta}} = \frac{1}{\xi^2} \int_{-\pi/2}^{\pi/2} \cos{\theta} \dd{\theta} = \frac{2}{\xi^2}
.\end{eqnarray}
Plugging this back into \eref{phi-half-plane}, we have
\begin{eqnarray}
    \Phi(\va*{x}) = \frac{V z}{\pi} \int_{-x}^{\infty} \frac{\dd{u}}{u^2 + z^2}
.\end{eqnarray}
Let us use the change of variables $u = z \tan{\phi}$ such that
\begin{align}
    \Phi(\va*{x}) &= \frac{Vz}{\pi} \int_{\arctan{-x/z}}^{\pi/2} \frac{z \sec^2{\phi} \dd{\phi}}{z^2 \sec^2{\phi}} = \frac{V}{\pi} \Big[ \frac{\pi}{2} + \arctan(x/z) \Big] \nonumber \\
    &= V \Big[ \frac{1}{2} + \frac{1}{\pi} \arctan(x/z) \Big]
.\end{align}

For the equipotential lines, we can insert polar coordinates (in the $xz$-plane) we have $x = s \tan{\phi}$ and $z = s \sin{\phi}$, which gives
\begin{eqnarray}
    \Phi(\va*{x}) = V \Big[ \frac{1}{2} + \frac{\phi}{\pi} \Big]
.\end{eqnarray}
Clearly, the equipotential surfaces are lines of constant $\phi$ with $z > 0$.

Furthermore, we can calculate the electric field as
\begin{align}
    \va*{E} = - \grad \Phi(\va*{x}) = - \vu*{\phi} \frac{1}{s} \pdv{\Phi(\va*{x})}{\phi} = - \frac{V}{\pi s}
,\end{align}
implying the surfaces of constant $\va*{E}$ are just half-cylinders for $z > 0$ along the $y$-axis.
Additionally, the surface charge density is then
\begin{eqnarray}
    \sigma = \epsilon_0 \va*{E} \cdot \vu*{n} = - \epsilon_0 \va*{E} \cdot \vu*{z} = \frac{V}{\pi \epsilon_0} \frac{1}{s}
.\end{eqnarray}

\section{Dirichlet Green Function for the Sphere}

Again, we want to calculate the potential for BCs on a spherical surface with radius $a$ with Diriclet BCs.
The method of images gives us the Green's function (outside the sphere) as
\begin{eqnarray}
    G_{D}(\va*{x},\va*{x}') = \frac{1}{|\va*{x}' - \va*{x}|} - \frac{a/x}{|\va*{x}' - (a/x)^2\va*{x}|}
.\end{eqnarray}
If we introduce the angle $\gamma$ between $\va*{x}$ and $\va*{x}'$, then
\begin{eqnarray}
    G_{D}(\va*{x},\va*{x}') = \frac{1}{\sqrt{x'^2 + x^2 - 2 x x' \cos{\gamma}}} - \frac{1}{\sqrt{x^2 x'^2/a^2 + a^2 - 2xx' \cos{\gamma}}}
.\end{eqnarray}
We then have
\begin{align}
    \pdv{G_{D}}{n'}\Big|_{x' = a} = - \pdv{G_{D}}{x'}\Big|_{x'=a} = - \frac{x^2 - a^2}{a(x^2 + a^2 - 2ax \cos{\gamma})^{3/2}}
.\end{align}
Note that we can rewrite $\cos{\gamma}$ in favor of $\theta$, $\theta'$, $\phi$, and $\phi'$ by using that
\begin{eqnarray}
    \cos{\gamma} = \frac{\va*{x} \cdot \va*{x}'}{|\va*{x}| |\va*{x}'|} = \vu*{n} \cdot \vu*{n}' = \sin{\theta} \sin{\theta'} \cos{(\phi - \phi')} + \cos{\theta} \cos{\theta'}
.\end{eqnarray}

Let us consider the example where the sphere has its hemispheres at equal and opposite potentials $V$.
\begin{align}
    \label{eq:general-pot-ang}
    \Phi(\va*{x}) &= -\frac{a^2}{4 \pi} \int_{0}^{2\pi} \dd{\phi'} \Bigg[ V \int_{0}^{\pi/2} \dd{\theta'} \sin{\theta'} \pdv{G_{D}}{n'} + (-V) \int_{\pi/2}^{\pi} \dd{\theta'} \sin{\theta'} \pdv{G_{D}}{n'} \Bigg]\nonumber \\
.\end{align}
If in the second integral, we make the substitution $\theta' \rightarrow \pi - \theta'$, we find
\begin{align}
    \Phi(\va*{x}) &= \frac{Va^2}{4 \pi} \int_{0}^{2\pi} \dd{\phi'} \int_{0}^{\pi/2} \dd{\theta'} \sin{\theta'} \Bigg[ \pdv{G}{n'} - \pdv{G}{n'}\Big|_{\theta' \rightarrow \pi - \theta'} \Bigg] \nonumber \\
    &= \frac{V}{4 \pi} a(x^2 - a^2) \int_{0}^{2 \pi} \dd{\phi'} \int_{0}^{1} \dd{(\cos{\theta'})} \Bigg[ \frac{1}{(x^2 + a^2 - 2ax \cos{\gamma})^{3/2}} - \frac{1}{(x^2 + a^2 + 2ax \cos{\gamma})^{3/2}} \Bigg]
.\end{align}
Observe that this is as far as we can go analytically in full generality.
We can, however, study some specific cases.

First, consider the solution on the $z$-axis with $\theta = 0$.
In this case, $\cos{\gamma} = \cos{\theta'}$ and $|\va*{x}| = z$, which gives (denoting $u = \cos{\theta'}$)
\begin{align}
    \label{eq:pot-z}
    \Phi(z) &= \frac{V}{4 \pi} a(z^2 - a^2) (2 \pi) \int_{0}^{1} \dd{u} \Bigg[ \frac{1}{(a^2 + z^2 - 2azu)^{3/2}} - \frac{1}{(a^2 + z^2 + 2azu)^{3/2}} \Bigg] \nonumber \\
    &= V \Bigg[ 1 - \frac{z^2 - a^2}{z \sqrt{z^2 + a^2}} \Bigg] \sim \frac{3 V}{2} \Big( \frac{a}{z} \Big)^2
,\end{align}
considering the limiting case $z \gg a$.

We can use \eref{general-pot-ang} to match this result in the large-distance limit where $x \gg a$.
Thus, writing 
\begin{eqnarray}
    a^2 + x^2 \pm 2ax\cos{\gamma} = (a^2 + x^2)(1 \pm 2\alpha \cos{\gamma})
,\end{eqnarray}
where we have defined $\alpha = ax/(x^2 + a^2)$.
Expanding,
\begin{align}
    [1 \pm 2 \alpha \cos{\gamma}]^{-3/2} &= 1 \pm \Big(-\frac{3}{2}\Big)(\mp 2 \alpha \cos{\gamma}) + \frac{1}{2!} \Big( -\frac{3}{2} \Big) \Big( -\frac{5}{2} \Big) (\mp 2 \alpha \cos{\gamma})^2 \nonumber \\
    &+ \frac{1}{3!} \Big( -\frac{3}{2} \Big) \Big( -\frac{5}{2} \Big) \Big( -\frac{7}{2} \Big) (\mp 2 \alpha \cos{\gamma})^3 + \ldots
.\end{align}
We thus have
\begin{align}
    \frac{1}{(1 - 2 \alpha \cos{\gamma})^{3/2}} - \frac{1}{(1 + 2 \alpha \cos{\gamma})^{3/2}} = 6 \alpha \cos{\gamma} + 35 \alpha^3 \cos^3{\gamma} + \ldots
,\end{align}
and
\begin{align}
    \Phi(\va*{x}) = \frac{V}{4 \pi} \frac{a(x^2 - a^2)}{x^2 + a^2} \int_{0}^{2\pi} \dd{\phi'} \int_{0}^{1} \dd{(\cos{\theta'})} [6 \alpha \cos{\gamma} + 35 \alpha^3 \cos^3{\gamma} + \ldots]
.\end{align}
{\color{red} After a bunch of algebra (to do later \ldots)}
\begin{align}
    \Phi(\va*{x}) = \frac{3 V a^2}{2 x^2} \Bigg[ \cos{\theta} - \frac{7a^2}{12 x^2} \Big( \frac{5}{2} \cos^3{\theta} - \frac{3}{2} \cos{\theta} \Big) + \ldots \Bigg]
,\end{align}
which agrees with \eref{pot-z} for $z \gg a$ at $\theta = 0$.


