\chapter{Method of Images}

\section{Charge Above an Infinite Grounded Plane}

The method of images is one of the more conventional methods that can be used to solve for the potential with certain setups.
Consider a charge a distance $a$ above a conductor spanning the $xy$-plane.
We are interested in the potential in the region $z > 0$, where Poisson's equation reads
\begin{eqnarray}
    \laplacian \Phi = -4 \pi q \delta^{(3)}(\va*{r} - \va*{a})
,\end{eqnarray}
where $\va*{a} = a\vu*{z}$.
The potential of the point charge is simple to write down, but on first glance, it does not seem a trivial matter how to write down the potential of the plane is at any arbitary point above the plane.

We can however be clever and place a so-called ``mirror charge'' $-q$ a distance $a$ below the surface of the plane.
The potential of this mirror setup is
\begin{align}
    \label{eq:generic-plane-pot}
    \Phi &= \frac{q}{4 \pi \epsilon_0} \Big( \frac{1}{|\va*{r} - \va*{a}|} - \frac{1}{|\va*{r} + \va*{a}|}\Big) \\
    \label{eq:specific-plane-pot}
         &- \frac{q}{4 \pi \epsilon_0} \Big( \frac{1}{\sqrt{s^2 + (z - a)^2}} - \frac{1}{\sqrt{s^2 + (z + a)^2}} \Big)
,\end{align}
where $s^2 = x^2 + y^2$.
From the form of \eref{specific-plane-pot}, it is clear that $\Phi(x,y,0) = 0$, meaning that the potential satisfies the BCs, and hence, this is the potential of the system since it satisfies Poisson's equation for $z > 0$:
\begin{eqnarray}
    \laplacian \Phi = -4 \pi q [ \delta(\va*{r} - \va*{a}) - \delta(\va*{r} + \va*{a}) ]
.\end{eqnarray}

\section{Point Charge Near a Grounded Sphere}

Consider a conducting, grounded sphere of radius $a$ with a charge $q$ placed a distance $b > a$ from the center of the sphere.
We are interested in the potential in the region $r > a$.
Again, it is not immediately clear how to write down the contribution to the potential from the conducting sphere, so we introduce an image charge $q'$ inside the sphere a distance $b'$ from the center along the axis defined by the line between the center and the source charge (which we call the $x$-axis).
The potential of the mirror setup is just
\begin{eqnarray}
    \Phi = \frac{1}{4 \pi \epsilon_0} \Bigg( \frac{q}{|\va*{r} - \va*{b}|} + \frac{q'}{|\va*{r} - \va*{b}'|} \Bigg)
,\end{eqnarray}
where $\va*{b} = b \vu*{x}$ and $\va*{b}' = b' \vu*{x}$

We can solve for $q'$ and $b'$ by imposing the BCs:
\begin{align}
    \Phi|_{x = a} &= \frac{1}{4 \pi \epsilon_0} \Big( \frac{q}{b - a} + \frac{q'}{a - b'} \Big) = 0 \\
    \Phi|_{x = -a} &= \frac{1}{4 \pi \epsilon_0} \Big( \frac{q}{a + b} + \frac{q'}{a + b'} \Big) = 0
.\end{align}
If we make some rearrangements, we have the following for $b'$:
\begin{gather}
    \frac{a + b'}{a - b'} = \frac{a + b}{b - a} \nonumber \\
    \Big( 1 + \frac{a + b}{b - a} \Big) b' = a \Big( \frac{a + b}{b - a} - 1 \Big) \nonumber \\
    \frac{2b}{b - a} b' = \frac{2a}{b - a} a \nonumber \\
    b' = \frac{a^2}{b}
.\end{gather}
Plugging this into one of the BCs
\begin{eqnarray}
    q' = -\frac{(a - a^2/b)}{b-a} q = -\frac{a}{b} q
.\end{eqnarray}
Thus, the potential of this setup is
\begin{align}
    \Phi = \frac{q}{4 \pi \epsilon_0} \Bigg( \frac{1}{|\va*{r} - \va*{b}|} - \frac{a/b}{|\va*{r} - (a^2/b^2)\va*{b}|} \Bigg)
.\end{align}

We can clearly see that this potential satisfies the Poisson equation since we only inserted mirror charges in the region $r < a$.
Checking the BC (on any arbitrary point on the conductor), notice that 
\begin{align}
    |\va*{a} - \va*{b}'|^2 = a^2 - 2a \frac{a^2}{b} \cos{\theta} = \frac{a^{4}}{b^2} = \frac{a^2}{b^2} \Big( b^2 - 2ab \cos{\theta} + a^2 \Big) = \Big( \frac{a}{b} \Big)^2 |\va*{a} - \va*{b}|
.\end{align}
Thus,
\begin{align}
    \Phi|_{r = a} &= \frac{q}{4 \pi \epsilon_0} \Bigg( \frac{1}{|\va*{a} - \va*{b}|} - \frac{a/b}{|\va*{a} - \va*{b}'|} \Bigg) = \frac{q}{4 \pi \epsilon_0} \Bigg( \frac{1}{|\va*{a} - \va*{b}|} - \frac{a/b}{(a/b)|\va*{a} - \va*{b}|} \Bigg) = 0
.\end{align}

Next, we can determine the induced charge density as
\begin{align}
    \sigma &= \epsilon_0 (\va*{E} \cdot \va*{r})_{r = a} = \frac{q}{4 \pi} \Bigg( \frac{\va*{a} - \va*{b}}{|\va*{a} - \va*{b}|^3} - \frac{(a/b)(\va*{a} - \va*{b}')}{|\va*{a} - \va*{b}'|^3} \Bigg) \cdot \vu*{r} \\
    &= \frac{q}{4 \pi} \Bigg( \frac{\va*{a} - \va*{b}}{|\va*{a} - \va*{b}|^3} - \frac{\va*{a} - (a/b)^2\va*{b}}{(a/b)^2|\va*{a} - \va*{b}|^3} \Bigg) \cdot \vu*{r} \\
    &= \frac{q}{4 \pi} \Bigg( \frac{\va*{a} - \va*{b}}{|\va*{a} - \va*{b}|^3} - \frac{(b/a)^2\va*{a} - \va*{b}}{|\va*{a} - \va*{b}|^3} \Bigg) \cdot \vu*{r} = - \frac{q}{4 \pi} \frac{ (b/a)^2 - 1 }{|\va*{a} - \va*{b}|^3} \va*{a} \cdot \vu*{r} \\
    &= - \frac{q}{4 \pi a} \frac{b^2 - a^2}{[a^2 + b^2 - 2ab\cos{\theta}]^{3/2}}
.\end{align}
One sanity check to be performed is that the total charge induced on the sphere is just $q'$:
\begin{align}
    Q = \int \sigma a^2 \dd{\Omega} = -\frac{q a}{2} (b^2 - a^2) \int_{0}^{\pi} \frac{\sin{\theta} \dd{\theta}}{[(a^2 + b^2) - 2ab \cos{\theta}]^{3/2}}
.\end{align}
Using the substitution $u = (a^2 + b^2) - 2ab \cos{\theta}$ ($\dd{u} = 2ab\sin{\theta} \dd{\theta}$), we have
\begin{align}
    Q &= -\frac{qa}{2} (b^2 - a^2) \frac{1}{2ab} \int_{(a-b)^2}^{(a+b)^2} u^{-3/2} \dd{u} \nonumber \\
    &= -\frac{q(b^2-a^2)}{4b} (-2) \Big[ \frac{1}{a + b} - \frac{1}{|a - b|} \Big] \nonumber \\
    &= \frac{q}{2b} [ (b-a) - (a + b) ] \nonumber \\
    &= -\frac{a}{b}q = q'
.\end{align}


\section{Point Charge Near an Isolated, Conducting Sphere at Potential $V$}

In this section, we change the problem of the last section slightly such that the potential is held to potential $V$ instead of being grounded.
This amounts to placing some charge on the sphere.
We can find the potential by placing an extra mirror charge $q''$ at the origin such that its potential on the sphere is $V$:
\begin{eqnarray}
    \Phi = \frac{1}{4 \pi \epsilon_0} \Bigg( \frac{q}{|\va*{r} - \va*{b}|} + \frac{q'}{|\va*{r} - \va*{b}'|} + \frac{q''}{r} \Bigg)
.\end{eqnarray}
We know that the first two terms satisfy the BCs for a grounded sphere, so we must have $q'' = 4 \pi \epsilon_0 Va$.
\begin{eqnarray}
    \Phi = \frac{1}{4 \pi \epsilon_0} \Bigg( \frac{q}{|\va*{r} - \va*{b}|} + \frac{q'}{|\va*{r} - \va*{b}'|} \Bigg) + \frac{a}{r} V
.\end{eqnarray}
From this, it is simple to identify the total charge on the sphere as
\begin{eqnarray}
    Q = q' + q''
,\end{eqnarray}
and if instead we were given the total charge $Q$ placed on the sphere, we would have
\begin{eqnarray}
    V = \frac{Q - q'}{4 \pi \epsilon_0 a}
.\end{eqnarray}
In this case then, the potential could be expressed as
\begin{eqnarray}
    \Phi = \frac{1}{4 \pi \epsilon_0} \Bigg( \frac{q}{|\va*{r} - \va*{b}|} + \frac{q'}{|\va*{r} - \va*{b}'|} + \frac{Q - q'}{r} \Bigg)
.\end{eqnarray}


\section{Force on charge $q$}

Finally, we can determine the force exerted on charge $q$ by the sphere as just that exerted by the mirror charges
\begin{align}
    \label{eq:force-mirror}
    \va*{F} &= q \va*{E}_{\rm mirror}(\va*{b}) = \frac{q}{4 \pi \epsilon_0} \Bigg( q'\frac{\va*{b} - \va*{b}'}{|\va*{b} - \va*{b}'|^3} + (Q - q') \frac{\va*{b}}{b^3} \Bigg) \nonumber \\
    &= \frac{q}{4 \pi \epsilon_0} \Bigg\{ \frac{q'}{[1 - (a/b)^2]^3 b^3}[1 - (a/b)^2]\va*{b} + (Q - q') \frac{\va*{b}}{b^3} \Bigg\} \nonumber \\
    &= \frac{q}{4 \pi \epsilon_0 b^3} \Bigg\{ Q - q' \Bigg( 1 - \frac{1}{[1 - (a/b)^2]^2} \Bigg) \Bigg\} \va*{b} \nonumber \\
    &= \frac{q}{4 \pi \epsilon_0} \frac{\va*{b}}{b^3} \Bigg\{ Q + q \Big( \frac{a}{b} \Big)^3 \frac{(a/b)^2 - 2}{[1 - (a/b)^2]^2} \Bigg\} \nonumber \\
    &= \frac{q}{4 \pi \epsilon_0} \frac{\va*{b}}{b^3} \Bigg\{ Q - \frac{q a^3(2b^2 - a^2)}{b(b^2 - a^2)^2} \Bigg\}
.\end{align}

It is somewhat difficult to extract physics from \eref{force-mirror} analytically.
Let us see what happens when we take $b \rightarrow a$:
\begin{eqnarray}
    \va*{F} \approx \frac{q}{4 \pi \epsilon_0} \frac{\vu*{b}}{a^2} \Bigg[ Q - \frac{q a^5}{a (2a)^2 (b-a)^2} \Bigg] \sim - \frac{q^2}{4 \pi \epsilon_0} \frac{1}{[2(b-a)]^2} \vu*{b}
.\end{eqnarray}
This mimics the form of the force between the source and mirror charges in the plane problem, which is interpreted to mean that when the source charge gets very close to the sphere, the surface becomes almost an infinite plane with a significant amount of induced charge that dominates the force on the source.


