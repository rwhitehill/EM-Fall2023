\chapter{Conservative Forces}

\section{Conservative Forces}

I called the electrostatic field in a previous section ``conservative''.
We will see why this is the case now.
Recall the definition of the work done on a particle under the influence of a force $\va*{F}$ over its motion along a curve $C$:
\begin{eqnarray}
    W = \int_{C} \va*{F}(\va*{r}) \cdot \dd{\va*{r}}
.\end{eqnarray}
The work done by an electric field $\va*{E}$ on a test charge is then
\begin{eqnarray}
    W = q \int_{C} \va*{E}(\va*{r}) \cdot \dd{\va*{r}}
,\end{eqnarray}
and since $\curl{\va*{E}} = 0$, we know from the equivalence theorem from above that if $C$ is a closed curve, then $W = 0$.
Furthermore, since $\va*{F} = q \va*{E}$, then $\curl{\va*{F}} = 0$, and we can therefore define a scalar potential (which we will see is just the electrostatic potential energy) $U(\va*{r})$ for the electric force such that $\va*{F} = - \grad U$, which is related to the electric potential by
\begin{eqnarray}
    U(\va*{r}) = q \Phi(\va*{r})
.\end{eqnarray}

Now, let us remind ourselves that the work done on a particle is given by the change in its kinetic energy.
Starting with Newton's 2nd law $m \ddot{\va*{r}} = \va*{F}(\va*{r}) = - \grad U(\va*{r})$, we can integrate this along a path, which gives
\begin{gather}
    \int_{\va*{a}}^{\va*{b}} m \ddot{\va*{r}} \cdot \dd{\va*{r}} = \int_{\va*{a}}^{\va*{b}} \va*{F} \cdot \dd{\va*{r}} = \int_{\va*{a}}^{\va*{b}} -\grad U(\va*{r}) \cdot \dd{\va*{r}} \nonumber \\
    W = \int_{\va*{a}}^{\va*{b}} \frac{1}{2} m \dv{\dot{\va*{r}}^2}{t} \dd{t} = \frac{1}{2} m \Delta \dot{\va*{r}}^2 = - \Delta U(\va*{r})
    \Rightarrow \Delta E = \Delta \Big( \frac{1}{2} m \dot{\va*{r}}^2 + U \Big) = 0
.\end{gather}
We define $E$ as the mechanical energy of a system and $U$ as the potential energy, and clearly under a conservative force $\va*{F}$, this is a conserved quantity.

\section{Potential Energy of a Charge Distribution}

If we consider two charges $q_1$ and $q_2$ at corresponding locations $\va*{r}_{1}$ and $\va*{r}_{2}$, then the potential energy of this system is given by
\begin{eqnarray}
    U_{12} = q_{1} E_{2}(\va*{r}_{1}) = \frac{1}{4 \pi \epsilon_0} \frac{q_1 q_2}{|\va*{r}_{1} - \va*{r}_{2}|}
.\end{eqnarray}
Notice that $U_{12} = U_{21}$, which should be the case.

If we generalize to a charge distribution with $N$ point charges, we then have the potential energy of the setup as 
\begin{align}
    U &= 0 + (U_{12}) + (U_{13} + U_{23}) + \ldots (U_{1N} + \ldots + U_{N-1,N}) \nonumber \\
               &= \sum_{i=2}^{N} \sum_{j = 1}^{i} U_{ij} 
.\end{align}
We can extend the second sum, noting that we are double counting
\begin{eqnarray}
    U = \frac{1}{2} \sum_{i} \sum_{j \ne i} U_{ij} = \frac{1}{2} \sum_{i} \sum_{j \ne i} q_{i} \Phi(\va*{r}_{j}) = \frac{1}{8 \pi \epsilon_0} \sum_{i} \sum_{j \ne i} \frac{q_{i} q_{j}}{|\va*{r}_{i} - \va*{r}_{j}|}
.\end{eqnarray}

If we further generalize to a continuous charge distribution, we have
\begin{align}
    U &= \frac{1}{2} \int \dd[3]{\va*{r}} \rho(\va*{r}) \Phi(\va*{r}) = \frac{\epsilon_0}{2} \int \dd[3]{\va*{r}} [\div{\va*{E}(\va*{r})}] \Phi(\va*{r}) \\
      &= - \frac{\epsilon_0}{2} \int \dd[3]{\va*{r}} \va*{E}(\va*{r}) \grad \Phi(\va*{r}) = \frac{\epsilon_0}{2} \int |\va*{E}(\va*{r})|^2 \dd[3]{\va*{r}}
.\end{align}
From this, we can define the electric energy density as $u_{E} = \epsilon_0 |\va*{E}|^2/2$.

If we imagine pulling off a very small surface element $\dd{S}$ a distance $\dd{x}$ away from the surface, the potential energy of the surface changes by 
\begin{eqnarray}
    \delta{U} = -u_{E} \dd{S} \dd{x} \Rightarrow \delta\va*{F} = -\dv{U}{x} = u_{E} \dd{\va*{S}}
.\end{eqnarray}
The total force on the surface is then found by integrating the energy density over the surface, which is given by
\begin{eqnarray}
    u_{E} = \frac{\sigma^2}{2 \epsilon_0}
,\end{eqnarray}
where $\sigma$ is the surface charge density on $S$.

We can derive this another way (Griffith's pg. 104).
The force on a patch of size $\dd{S}$ is
\begin{eqnarray}
    \dd{\va*{F}} = \va*{E} \dd{q} = \sigma \va*{E} \dd{S}
.\end{eqnarray}
Note that $\va*{E}$ is the electric field attributable to all points at the surface element $\dd{S}$ except for the patch itself.
We know in general that the electric field is discontinous at the surface, and this discontinuity is exactly attributable to the surface charge $\sigma \dd{S}$ via $(\va*{E}_{\rm out} - \va*{E}_{\rm in}) \cdot \vu*{n}= \sigma/\epsilon_0$.
At this point, note that $\va*{E}_{\rm out} = \va*{E} + \va*{E}_{\rm patch}$ and $\va*{E}_{\rm in} = \va*{E} - \va*{E}_{\rm patch}$, giving
\begin{eqnarray}
    \va*{E}_{\rm patch} \cdot \vu*{n} = \frac{\sigma}{2 \epsilon_0}
,\end{eqnarray}
and therefore,
\begin{eqnarray}
    \va*{E} = \frac{1}{2} (\va*{E}_{\rm out} + \va*{E}_{\rm in})
,\end{eqnarray}
which is just the average of the fields inside and outside the volume bounded by $S$.
If we specialize to conductors, then we know $\va*{E}_{\rm in} = 0$ and $\va*{E}_{\rm out} = (\sigma/\epsilon_0) \vu*{n}$, which gives
\begin{eqnarray}
    \va*{E} = \frac{\sigma}{2 \epsilon_0}
,\end{eqnarray}
and 
\begin{eqnarray}
    \dd{\va*{F}} = \frac{\sigma^2}{2 \epsilon_0} \dd{\va*{S}}
.\end{eqnarray} 



