\chapter{Laplace's and Poisson's Equations}

As a general wisdom, it is easier to deal with the electric potential than the field directly since it is just one function as opposed to three different components.
To that end, we derive a differential equation for $\Phi$ as follows:
\begin{eqnarray}
    \div{\va*{E}(\va*{r})} = \div{-\grad \Phi(\va*{r})} = -\laplacian \Phi(\va*{r}) = \frac{\rho(\va*{r})}{\epsilon_0}
.\end{eqnarray}
This is Poisson's equation.
If there are no sources, it reduces to Laplace's equation
\begin{eqnarray}
    \laplacian \Phi(\va*{r}) = 0
.\end{eqnarray}

As general features, these are both linear, second order partial differential equations for the scalar potential.
As such, we require some boundary conditions to specify the value of $\Phi$ (or its derivative) on some surface in order to uniquely determine its form.

If we reduce the problem to one-dimension, Poisson's equation is written as
\begin{eqnarray}
    \pdv[2]{\Phi}{x} = \lambda
,\end{eqnarray}
and has solution $\Phi(x) = \lambda x^2 / 2 + Ax + B$, where $A$ and $B$ are determined by boundary conditions at some points $x_1$ and $x_2$.

\textbf{(1)}: Dirichlet BCs simply specify the potential's value on the surface $S$:
\begin{eqnarray}
    \Phi(\va*{r})|_{\va*{r} \in S} = f(\va*{r})
.\end{eqnarray}
For example, we may hold a conductor to ground, which specifies $\Phi \equiv 0$ on the surface of the conductor.

\textbf{(2)}: Neumann BCs specify the normal derivative of the potential on the surface $S$ as 
\begin{eqnarray}
    \vu*{n} \cdot \grad \Phi = \pdv{\Phi}{n}\Big|_{\va*{r} \in S} = g(\va*{r})
.\end{eqnarray}
For example, we may have a situtation where there is a charge on the surface $S$, which defines the electric field and therefore the gradient of the potential on $S$.








