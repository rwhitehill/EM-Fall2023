\chapter{Coulomb's Law}

In this section we deal with electrostatics, which is the idealization where the charge configurations are stationary in time, or rather, we only deal with their instantaneous properties.
For electrostatics, the governing equation is Coulomb's law (we use traditional SI units):
\begin{eqnarray}
    \va*{F}_{12} = \frac{q_1 q_2}{4 \pi \epsilon_0} \frac{\va*{r}_{1} - \va*{r}_{2}}{|\va*{r}_{1} - \va*{r}_{2}|^3}
,\end{eqnarray}
where $\epsilon_0 = 8.854 \times 10^{-12}~{\rm C^2 N^{-1} m^{-2}}$ is the permittivity of free space.
This is the force on charge $1$ by charge $2$.
Notice that from Newton's 3rd law, the force $\va*{F}_{21} = -\va*{F}_{12}$ (equal and opposite).

Coulomb's law is an empirical observation.
That is, the inverse square law is written down to match experimental observations, and indeed, it does a fantastic job explaining electrostatic phenomena.
We will see that it is incomplete for a dynamic picture of electromagnetism, but for now (the next several lectures), it will be all we need.

Another empirical observation is that the force on a charge $q$ by a configuration of charges $q_{i}$ at positions $\va*{r}_{i}$, respectively, is just a linear superposition of the individual forces on $q$ by $q_{i}$:
\begin{eqnarray}
    \va*{F} = q \Bigg[ \frac{1}{4 \pi \epsilon_0} \sum_{i} q_{i} \frac{\va*{r} - \va*{r}_{i}}{|\va*{r} - \va*{r}_{i}|^3} \Bigg] = q \va*{E}(\va*{r})
.\end{eqnarray}
The quantity $\va*{E}$ is defined as the electrostatic field due to the charge configuration.
It's rigorous definition requires a bit more care, actually, since the electric field due to $q$ would distort the electric field due to the configuration, so
\begin{eqnarray}
    \va*{E}(\va*{r}) = \lim_{q \rightarrow 0} \frac{\va*{F}(\va*{r})}{q}
.\end{eqnarray}

In the above definitions we dealt with discrete point charges, but in general, we may deal with continuous charge distributions.
Charge distributions are idealizations of course. 
Realistically speaking, charge is quantized in units of $e = 1.602 \times 10^{-19}~{\rm C}$, but if we have a large number of charges then the total charge of a distribution is $Q = (N_{+} - N_{-})e = \int \dd{q} = \int \dd[3]{\va*{r}} \rho(\va*{r})$ in the limit $e/Q \rightarrow 0$, where $N_{\pm}$ is the number of units of positive/negative charge in the solid body.
The function $\rho(\va*{r})$ is the volume charge density.
From this, we can then write the electrostatic field of a continuous charge distribution with charge density $\rho(\va*{r})$ as
\begin{eqnarray}
    \va*{E}(\va*{r}) = \frac{1}{4 \pi \epsilon_0} \int \dd[3]{\va*{r}'} \rho(\va*{r}') \frac{\va*{r} - \va*{r}'}{|\va*{r} - \va*{r}'|^3}
.\end{eqnarray}


