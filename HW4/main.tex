\def\duedate{\today}
\def\HWnum{4}
\input{../preamble.tex}

\begin{document}

\prob{1}{

    Consider a potential problem in the half-space defined by $z \geq 0$, with Dirichlet boundary conditions on the plane $z = 0$ (and at infinity). \\[1pt]

(a) Write down the appropriate Green function $G(\va*{x},\va*{x}')$.

(b) If the potential on the plane $z = 0$ is specified to be $\Phi = V$ inside a circle of radius $a$ centered at the origin, and $\Phi = 0$ outside that circle, find an integral expression for the potential at the point $P$ specified in terms of cylindrical coordinates $(\rho,\phi,z)$.

(c) Show that, along the axis of the circle ($\rho = 0$), the potential is given by
\begin{eqnarray}
    \Phi = V \Big( 1 - \frac{z}{\sqrt{ a^2 + z^2}} \Big)
.\end{eqnarray}

(d) Show that at large distances ($\rho^2 + z^2 \gg a^2$) the potential can be expanded in a power series in $(\rho^2 + z^2)^{-1}$, and that the leading terms are
\begin{eqnarray}
    \Phi = \frac{Va^2}{2}\frac{z}{(\rho^2 + z^2)^{3/2}}  \Big[ 1 - \frac{3 a^2}{4 (\rho^2 + z^2)} + \frac{5( 3 \rho^2 a^2 + a^{4} )}{8( \rho^2 + z^2 )^2} + \ldots \Big]
.\end{eqnarray}
Verify that the results of parts (c) and (d) are consistent with each other in their common range of validity.

}

\sol{

(a) A Green's function satisfying
\begin{eqnarray}
    \laplacian G(\va*{x},\va*{x}') = -4 \pi \delta(\va*{x} - \va*{x}')
\end{eqnarray}
is $G(\va*{x},\va*{x}') = |\va*{x} - \va*{x}'|^{-1}$.
Now, we must find a Green's function satisfying Dirichlet boundary conditions.
That is, we take $G \rightarrow G + F$, where $F$ solves Laplace's equation (in the half-space where $z \geq 0$) and $G + F$ satisfies the boundary condition $\Phi = 0$ when $z = 0$.
Such a choice is
\begin{eqnarray}
    \eqbox{ G(\va*{x},\va*{x}') = \frac{1}{|\va*{x}' - \va*{x}|} - \frac{1}{|\va*{x}' - \va*{y}|} }
,\end{eqnarray}
where $\va*{y}$ is $\va*{x}$ translated over the $xy$-plane (i.e. $z \rightarrow -z$ for $\va*{y}$).
It is clear then that the second term satisfies Laplaces equation for $z \geq 0$ and that the sum of these two functions is identically zero on the $xy$-plane.

(b) If we have the potential on the plane
\begin{eqnarray}
    \Phi(x,y,z = 0) = \begin{cases}
        V & x^2 + y^2 \leq a^2 \\
        0 & {\rm otherwise}
    ,\end{cases}
\end{eqnarray}
we can write the potential for $z \geq 0$ as
\begin{eqnarray}
    \Phi = \frac{1}{4 \pi \epsilon_0} \int_{V} \dd[3]{\va*{x}'} G(\va*{x},\va*{x}') \rho(\va*{x}') - \frac{1}{4 \pi} \int_{S} \dd{S'} \Phi(\va*{x}') \pdv{G(\va*{x},\va*{x}')}{n'}
.\end{eqnarray}
Observe that the first term is zero since there is no charge.
The second integral is over the circular surface of radius $a$ in the $xy$ plane such that
\begin{eqnarray}
    \begin{aligned}
        \Phi &= -\frac{V}{4 \pi} \int_{0}^{a} \int_{0}^{2\pi} \dd{\rho'} \dd{\phi'} \rho' \Big( - \pdv{G}{z'} \Big) \\
        &= \frac{V}{4 \pi} \int_{0}^{a} \int_{0}^{2 \pi} \dd{\rho'} \dd{\phi'} \rho' \frac{2z}{[ (x' - x)^2 + (y' - y)^2 + z^2 ]^{3/2}} \\
        &= \eqbox{ \frac{V z}{2 \pi} \int_{0}^{a} \int_{0}^{2 \pi} \dd{\rho'} \dd{\phi'} \frac{\rho'}{[(\rho'\cos{\phi'} - \rho \cos{\phi})^2 + (\rho' \sin{\phi'} - \rho \sin{\phi})^2 + z^2]^{3/2}} }
    .\end{aligned}
\end{eqnarray}
At this point, this is all we can do in full generality.

(c) We consider a special case along the $z$ axis such that $\rho = 0$, which causes the previous expression to simply as
\begin{eqnarray}
    \eqbox{
    \begin{aligned}
        \Phi &= \frac{V z}{2 \pi} \int_{0}^{a} \int_{0}^{2 \pi} \dd{\rho'} \dd{\phi'} \frac{\rho'}{(\rho'^2 + z^2)^{3/2}} = V z \int_{0}^{a} \frac{\rho' \dd{\rho'}}{(\rho'^2 + z^2)^{3/2}} \\
        &= Vz \Bigg[ \frac{1}{z} - \frac{1}{\sqrt{z^2 + a^2}} \Bigg] = V \Bigg[ 1 - \frac{z}{\sqrt{z^2 + a^2}} \Bigg]
    \end{aligned}
    }
\end{eqnarray}
as desired.

(d) There is another case in which we can do the integration, which is when we are far away from circular surface with radius $a$ in the $xy$-plane (i.e. $\rho^2 + z^2 \gg a^2$).
In this case, we can write
\begin{eqnarray}
    \label{eq:prob1c}
    \Phi = \frac{V z}{2 \pi} \frac{1}{(\rho^2 + z^2)^{3/2}} \int_{0}^{a} \int_{0}^{2 \pi} \dd{\rho'} \dd{\phi'} \rho' \Bigg[ 1 + \frac{\rho'^2 - 2 \rho' \rho \cos{(\phi' - \phi)}}{\rho^2 + z^2} \Bigg]^{-3/2}
.\end{eqnarray}
Notice that the expression in brackets is of the form $(1 + x)^{n} = 1 + nx + [n(n-1)/2!] x^2 + \ldots $, so the double integral becomes
\begin{eqnarray}
    \begin{aligned}
        \int_{0}^{a} \int_{0}^{2 \pi} \dd{\rho'} \dd{\phi'} \rho' &\Bigg[ 1  - \frac{3}{2} \Big( \frac{\rho'^2 - 2 \rho' \rho \cos{(\phi' - \phi)}}{\rho^2 + z^2} \Big) + \frac{15}{8} \Big( \frac{\rho'^2 - 2 \rho' \rho \cos{(\phi' - \phi)}}{\rho^2 + z^2} \Big)^2 + \ldots \Bigg] \\
        &= \pi a^2 - \frac{1}{\rho^2 + z^2} \frac{3}{2} \frac{2 \pi}{4} a^{4} + \frac{1}{(\rho^2 + z^2)^{2}} \frac{15}{8} \Big[ \frac{2 \pi}{6} a^{6} + \frac{4 \pi}{4} \rho^2 a^4 \Big] + \ldots \\
        &= \pi a^2 \Big[ 1 - \frac{3 a^2}{4(\rho^2 + z^2)} + \frac{5(3 \rho^2 a^2 + a^{4})}{8(\rho^2 + z^2)^2} + \ldots \Big]
    .\end{aligned}
\end{eqnarray}
Plugging this back into \eref{prob1c}, we find
\begin{eqnarray}
   \eqbox{
       \Phi = \frac{V a^2}{2} \frac{z}{(\rho^2 + z^2)^{3/2}} \Big[ 1 - \frac{3 a^2}{4(\rho^2 + z^2)} + \frac{5(3 \rho^2 a^2 + a^{4})}{8(\rho^2 + z^2)^2} + \ldots \Big]
   } 
.\end{eqnarray}

(d) One sanity check to perform is that the result of part (c) reduces to that of part (b) in the limit $z \gg a$, where $\rho = 0$.
Observe that part (c) gives
\begin{eqnarray}
    \label{eq:prob1-d1}
    \Phi = \frac{V}{2} \Big( \frac{a}{z} \Big)^2 \Big[ 1 - \frac{3}{4} \Big( \frac{a}{z} \Big)^2 + \frac{5}{8} \Big( \frac{a}{z} \Big)^{4} + \ldots \Big] 
,\end{eqnarray}
and expanding the result of part (b) in powers of $\epsilon = a/z$, we have
\begin{eqnarray}
    \label{eq:prob1-d2}
    \begin{aligned}
        \Phi &= V \Big[ 1 - (1 + \epsilon^2)^{-1/2} \Big] = V \Big( \frac{1}{2} \epsilon^2 - \frac{3}{8} \epsilon^4 + \frac{5}{16} \epsilon^6 + \ldots \Big) \\
        &= \frac{V}{2} \epsilon^2 \Big( 1 - \frac{3}{4} \epsilon^2 + \frac{5}{8} \epsilon^{4} + \ldots \Big)
    .\end{aligned}
\end{eqnarray}
One can see that \eref{prob1-d1} and \eref{prob1-d2} match as needed.


}


\prob{2}{

    A two-dimensional potential problem is defined by two straight parallel line charges separated by a distance $R$ with equal and opposite linear charge densities $\lambda$ and $-\lambda$ \\[1pt]

(a) Show by direct construction that the surface of constant potential $V$ is a circular cylinder (circle in the transverse dimensions) and find the coordinates of the axis of the cylinder and its radius in terms of $R$, $\lambda$, and $V$.

(b) Use the results of part (a) to show that the capacitance per unit length $C$ of two right-circular cylindrical conductors, with radii $a$ and $b$ separated by a distance $d > a + b$, is
\begin{eqnarray}
    C = \frac{2 \pi \epsilon_0}{\cosh^{-1}\big(\frac{d^2 - a^2 - b^2}{2 a b}\big)} 
.\end{eqnarray}

(c) Verify that the result for $C$ agrees with the answer in Problem 1.7 of \textit{Jackson textbook} in the appropriate limit and determine the next nonvanishing order correction in powers of $a/d$ and $b/d$.

(d) Repeat the calculation of the capacitance per unit length for two cylinders inside each other ($d < |b - a|$).
Check the result for concentric cylinders ($d = 0$).

}

\begin{figure}[h!]
    \centering
    %\includegraphics[width=]{}
    \caption{}
    \label{fig:prob2a}
\end{figure}

\sol{

The potential a distance $\rho$ from a line charge with linear charge density $\lambda$ is
\begin{eqnarray}
    \Phi = \frac{\lambda}{2 \pi \epsilon_0} \ln{\Big( \frac{\rho_0}{\rho} \Big)} 
,\end{eqnarray}
where $\rho_0$ is some reference distance from the line charge where $\Phi = 0$.
The potential of a configuration of two parallel, oppositely charged lines is just
\begin{eqnarray}
    \Phi = \Phi_{+} + \Phi_{-} = \frac{\lambda}{2 \pi \epsilon_0} \ln{ \Big( \frac{\rho_{-}}{\rho_{+}} \Big) }
,\end{eqnarray}
where $\rho_{\pm}$ is just the distance between the point at which the potential is being evaluated and the line with charge per unit length $\pm \lambda$.
Notice that the dependence on $\rho_0$ cancels.
If we set up our coordinate system as in \fref{prob2a}, then we can express
\begin{eqnarray}
    \rho_{\pm} = \sqrt{ \Big( \frac{R}{2} \Big)^2 + r^2 \pm 2 \Big( \frac{R}{2} \Big) r \cos{\phi}}
.\end{eqnarray}
Hence, the potential in our coordinate system is
\begin{eqnarray}
    \Phi = \frac{\lambda}{4 \pi \epsilon_0} \ln{ \Big( \frac{R^2 + 4 r^2 - 4 R r \cos{\phi}}{R^2 + 4 r^2 + 4 R r \cos{\phi} } \Big) }
.\end{eqnarray}
On the surface of constant potential $\Phi = V$ we have
\begin{gather}
        e^{ 4 \pi \epsilon_0 V / \lambda } = \frac{R^2 + 4 r^2 - 4 R r \cos{\phi}}{R^2 + 4 r^2 + 4 R r \cos{\phi}} \\
        a(R^2 + 4r^2 + 4 R r \cos{\phi}) = R^2 + 4 r^2 - 4 R r \cos{\phi} \\
        (a - 1) R^2 + 4(a - 1) r^2 + 4(a + 1) R r \cos{\phi} = 0
\end{gather}
where we have denoted $a = \exp(4 \pi \epsilon_0 V / \lambda)$.
It is difficult to see a resemblance to any familiar surface in this form, so let us transform back into Cartesian coordinates, where $x = r \cos{\phi}$ and $y = r \sin{\phi}$:
\begin{gather}
    (a - 1) R^2 + 4(a - 1) (x^2 + y^2) + 4(a + 1) R x = 0 \\
    x^2 + \frac{a + 1}{a-1}Rx + y^2 = - \frac{R^2}{4} \\
    \Big[ x + \frac{a+1}{a-1} \frac{R}{2} \Big]^2 + y^2 = - \frac{R^2}{4} + \Big(\frac{a+1}{a - 1} \Big)^2  \frac{R^2}{4} \\
    \Big[ x + \coth{\Big( \frac{2 \pi \epsilon_0 V}{\lambda} \Big)} \frac{R}{2} \Big]^2 + y^2 = \Big[ \coth^2{\Big( \frac{2 \pi \epsilon_0 V}{\lambda} \Big)} - 1 \Big] \Big( \frac{R}{2} \Big)^2 \\
    \Big[ x + \coth{\Big( \frac{2 \pi \epsilon_0 V}{\lambda} \Big)} \frac{R}{2} \Big]^2 + y^2 = \frac{(R/2)^2}{\sinh^2{( 2 \pi \epsilon_0 V / \lambda ) }}
.\end{gather}
This is just the equation of a circle with center $(-\coth{(2 \pi \epsilon_0 V / \lambda)} [R / 2], 0)$ and radius $(R/2)/|\sinh{(2 \pi \epsilon_0 V / \lambda)}|$.
Notice that if $V > 0$ that the center of this circle is always 
Of course, the equipotential surface is a cylinder since the problem is translation invariant parallel to the lines.

(b) The capacitance of a setup with two conductors, one with charge $Q$ and the other with charge $-Q$, is just $C = Q/V$, where $V$ is the potential difference between the conductors.
In this problem, we have two cylinderical conductors, with radii $a$ and $b$, respectively.
We know that conductors are equipotential surfaces.
Hence, we can treat the potential on these surfaces as being set up by two line charges separated by some distance.

}


\prob{3}{

An insulated, spherical, conducting shell of radius $a$ is in a uniform electric field $E_0$.
If the sphere is cut into two hemispheres by a plane perpendicular to the field, find the force required to prevent the hemispheres from separating \\[1pt]

(a) if the shell is uncharged;

(b) if the total charge on the shell is $Q$.

}


\end{document}
