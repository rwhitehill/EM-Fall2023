\def\duedate{\today}
\def\HWnum{2}
\documentclass[10pt,a4paper]{book}

% custom section formatting
\usepackage{titlesec}
\titleformat{\chapter}[display]
{\normalfont\Large\filcenter\sffamily}
{\titlerule[1pt]%
\vspace{1pt}%
\titlerule
\vspace{1pc}%
\LARGE\MakeUppercase{\chaptertitlename} \thechapter}
{1pc}
{\titlerule
\vspace{1pc}%
\Huge}

% appendix handling
\usepackage[toc,page]{appendix}
    
% encoding for file and font
\usepackage[utf8]{inputenc}
\usepackage[T1]{fontenc}

% math formatting/tools
\usepackage{amsmath}
\usepackage{amssymb}
\usepackage{mathtools}
\usepackage[arrowdel]{physics}

% unit formatting
\usepackage{siunitx}
\AtBeginDocument{\RenewCommandCopy\qty\SI}

% figure formatting/tools
\usepackage{graphicx}
\usepackage{float}
\usepackage{subcaption}
\usepackage{multirow}
\usepackage{import}
\usepackage{pdfpages}
\usepackage{transparent}
\usepackage{currfile}

\NewDocumentCommand\incfig{O{1} m}{
    \def\svgwidth{#1\textwidth}
    \import{./Figures/\currfiledir}{#2.pdf_tex}
}

\newcommand{\bef}{\begin{figure}[h!tb]\centering}
\newcommand{\eef}{\end{figure}}

\newcommand{\bet}{\begin{table}[h!tb]\centering}
\newcommand{\eet}{\end{table}}

% hyperlink references 
\usepackage{hyperref}
\hypersetup{
    colorlinks=true,
    linkcolor=blue,
    filecolor=magenta,
    urlcolor=cyan,
    pdftitle={Physics 1 Notes},
    pdfauthor={Richard Whitehill},
    pdfpagemode=FullScreen
}
\urlstyle{same}

\newcommand{\eref}[1]{Eq.~(\ref{eq:#1})}
\newcommand{\erefs}[2]{Eqs.~(\ref{eq:#1})--(\ref{eq:#2})}

\newcommand{\fref}[1]{Fig.~(\ref{fig:#1})}
\newcommand{\frefs}[2]{Fig.~(\ref{fig:#1})--(\ref{fig:#2})}

\newcommand{\aref}[1]{Appendix~(\ref{app:#1})}
\newcommand{\sref}[1]{Section~(\ref{sec:#1})}
\newcommand{\srefs}[2]{Sections~(\ref{sec:#1})-(\ref{sec:#2})}

\newcommand{\tref}[1]{Table~(\ref{tab:#1})}
\newcommand{\trefs}[2]{Table~(\ref{tab:#1})--(\ref{tab:#2})}

% tcolorbox formatting/definitions
\usepackage[most]{tcolorbox}
\usepackage{xcolor}
\usepackage{xifthen}
\usepackage{parskip}

\definecolor{peach}{rgb}{1.0,0.8,0.64}

\DeclareTColorBox[auto counter, number within=chapter]{defbox}{O{}}{
    enhanced,
    boxrule=0pt,
    frame hidden,
    borderline west={4pt}{0pt}{green!50!black},
    colback=green!5,
    before upper=\textbf{Definition \thetcbcounter \ifthenelse{\isempty{#1}}{}{: #1} \\ },
    sharp corners
}

\newcommand*{\eqbox}{\tcboxmath[
    enhanced,
    colback=black!10!white,
    colframe=black,
    sharp corners,
    size=fbox,
    boxsep=8pt,
    boxrule=1pt
]}

\newtcolorbox[auto counter, number within=chapter]{exbox}{
    parbox=false,
    breakable,
    enhanced,
    sharp corners,
    boxrule=1pt,
    colback=white,
    colframe=black,
    before upper= \textbf{Example \thetcbcounter:}\,,
    before lower= \textbf{Solution:}\,,
    segmentation hidden
}

\newtcolorbox{resbox}{
    enhanced,
    colback=black!10!white,
    colframe=black,
    boxrule=1pt,
    boxsep=0pt,
    top=2pt,
    ams nodisplayskip,
    sharp corners
}


\begin{document}

\prob{1}{

Using Gauss' theorem we found that, in cylindrical coordinates $(\rho, \phi,z)$, the electric field of an infinitely long, thin rod carrying charge $\lambda$ per unit length is given by
\begin{eqnarray}
    \va*{E}(\rho) = \frac{\lambda}{2 \pi \epsilon_{0}} \frac{\vu*{e}_{\rho}}{\rho}
.\end{eqnarray}
Obtain this result by a direct integration using the formula
\begin{eqnarray}
    \va*{E}(\va*{r}) = \frac{1}{3 \pi \epsilon_0} \int_{V} \rho(\va*{r}') \frac{\va*{r} - \va*{r}'}{|\va*{r} - \va*{r}'|^3} \dd{V'}
.\end{eqnarray}
Similarly, using 
\begin{eqnarray}
    \Phi(\va*{r}) = \frac{1}{4\pi\epsilon_0} \int_{V} \frac{\rho(\va*{r}')}{|\va*{r} - \va*{r}'|} \dd{V'}
,\end{eqnarray}
find the electric potential corresponding to this field.
Show that $\Phi(\va*{r}) \equiv \Phi(\rho,\phi,z)$ depends in this case on $\rho$ only, and assume that $\Phi(\rho = \rho_0) = 0$ to fix additive constant.

}

\sol{}


\prob{2}{

    Each of the three charged spheres of radius $a$, one conducting , one having a uniform charge density within its volume, and one having a spherically symmetric charge density that varies radially as $r^{n}~(n > -3)$, has a total charge $Q$.
Use Gauss' theorem to obtain the electric fields both inside and outside each sphere.
Sketch the behavior of the fields as a function of radius for the first two spheres, and for the third with $n = \pm 2$.

}

\sol{}


\prob{3}{

The time-averaged potential of a neutral hydrogen atom is given by
\begin{eqnarray}
   \Phi = \frac{q}{4 \pi \epsilon_0} \frac{e^{-\alpha r}}{r} \Big( 1 + \frac{\alpha r}{2} \Big)
,\end{eqnarray}
where $q$ is the magnitude of the electronic charge, and $\alpha^{-1} = a_0/2$, $a_0$ being the Bohr radius.
Find the distribution of charge (both continuous and discrete) that will give this potential and interpret your result physically.

}

\sol{}


\prob{4}{

A simple capacitor is a device formed by two insulated conductors adjacent to each other.
If equal and opposite charges are placed on the conductors, there will be a certain difference of potential between them.
The ratio of the magnitude of the charge on one conductor to the magnitude of the potential ifference is called the capacitance (in SI units it is measured in farads).
Using Gauss' law, calculate the capacitance of \\[3pt]

(a) two large, flat, conducting sheets of area $A$, separated by a small distance $d$; \\[3pt]

(b) two concentric conducting spheres with radii $a,b$ ($b > a$); \\[3pt]

(c) two concentric conducting cylinders of length $L$, large compared to their radii $a,b$ ($b > a$).

}

\sol{}


\prob{5}{

    (a) For the three capacitor geometries in Problem 4 calculate the total electrostatic energy and express it alternatively in terms of the equal and opposite charges $Q$ and $-Q$ placed on the conductors and the potential difference between them. \\[3pt]

(b) Sketch the energy density of the electrostatic field in each case as a function of the appropriate linear corrdinate.

}

\sol{}


\prob{6}{

    Calculate the attractive force between conductors in the parallel plate capacitor (Problem 4(a)) for \\[3pt]

(a) fixed charges on each conductor; \\[3pt]

(b) fixed potential difference between conductors.

}



\end{document}
