\def\duedate{\today}
\def\HWnum{9}
\documentclass[10pt,a4paper]{book}

% custom section formatting
\usepackage{titlesec}
\titleformat{\chapter}[display]
{\normalfont\Large\filcenter\sffamily}
{\titlerule[1pt]%
\vspace{1pt}%
\titlerule
\vspace{1pc}%
\LARGE\MakeUppercase{\chaptertitlename} \thechapter}
{1pc}
{\titlerule
\vspace{1pc}%
\Huge}

% appendix handling
\usepackage[toc,page]{appendix}
    
% encoding for file and font
\usepackage[utf8]{inputenc}
\usepackage[T1]{fontenc}

% math formatting/tools
\usepackage{amsmath}
\usepackage{amssymb}
\usepackage{mathtools}
\usepackage[arrowdel]{physics}

% unit formatting
\usepackage{siunitx}
\AtBeginDocument{\RenewCommandCopy\qty\SI}

% figure formatting/tools
\usepackage{graphicx}
\usepackage{float}
\usepackage{subcaption}
\usepackage{multirow}
\usepackage{import}
\usepackage{pdfpages}
\usepackage{transparent}
\usepackage{currfile}

\NewDocumentCommand\incfig{O{1} m}{
    \def\svgwidth{#1\textwidth}
    \import{./Figures/\currfiledir}{#2.pdf_tex}
}

\newcommand{\bef}{\begin{figure}[h!tb]\centering}
\newcommand{\eef}{\end{figure}}

\newcommand{\bet}{\begin{table}[h!tb]\centering}
\newcommand{\eet}{\end{table}}

% hyperlink references 
\usepackage{hyperref}
\hypersetup{
    colorlinks=true,
    linkcolor=blue,
    filecolor=magenta,
    urlcolor=cyan,
    pdftitle={Physics 1 Notes},
    pdfauthor={Richard Whitehill},
    pdfpagemode=FullScreen
}
\urlstyle{same}

\newcommand{\eref}[1]{Eq.~(\ref{eq:#1})}
\newcommand{\erefs}[2]{Eqs.~(\ref{eq:#1})--(\ref{eq:#2})}

\newcommand{\fref}[1]{Fig.~(\ref{fig:#1})}
\newcommand{\frefs}[2]{Fig.~(\ref{fig:#1})--(\ref{fig:#2})}

\newcommand{\aref}[1]{Appendix~(\ref{app:#1})}
\newcommand{\sref}[1]{Section~(\ref{sec:#1})}
\newcommand{\srefs}[2]{Sections~(\ref{sec:#1})-(\ref{sec:#2})}

\newcommand{\tref}[1]{Table~(\ref{tab:#1})}
\newcommand{\trefs}[2]{Table~(\ref{tab:#1})--(\ref{tab:#2})}

% tcolorbox formatting/definitions
\usepackage[most]{tcolorbox}
\usepackage{xcolor}
\usepackage{xifthen}
\usepackage{parskip}

\definecolor{peach}{rgb}{1.0,0.8,0.64}

\DeclareTColorBox[auto counter, number within=chapter]{defbox}{O{}}{
    enhanced,
    boxrule=0pt,
    frame hidden,
    borderline west={4pt}{0pt}{green!50!black},
    colback=green!5,
    before upper=\textbf{Definition \thetcbcounter \ifthenelse{\isempty{#1}}{}{: #1} \\ },
    sharp corners
}

\newcommand*{\eqbox}{\tcboxmath[
    enhanced,
    colback=black!10!white,
    colframe=black,
    sharp corners,
    size=fbox,
    boxsep=8pt,
    boxrule=1pt
]}

\newtcolorbox[auto counter, number within=chapter]{exbox}{
    parbox=false,
    breakable,
    enhanced,
    sharp corners,
    boxrule=1pt,
    colback=white,
    colframe=black,
    before upper= \textbf{Example \thetcbcounter:}\,,
    before lower= \textbf{Solution:}\,,
    segmentation hidden
}

\newtcolorbox{resbox}{
    enhanced,
    colback=black!10!white,
    colframe=black,
    boxrule=1pt,
    boxsep=0pt,
    top=2pt,
    ams nodisplayskip,
    sharp corners
}


\begin{document}

\prob{1}{

A localized distribution of charge has a charge density
\begin{eqnarray}
    \rho(\va*{r}) = \frac{1}{64 \pi} r^2 e^{-r} \sin^2{\theta}
.\end{eqnarray}

(a) Make a multipole expansion of the potential due to this charge density and determine all the nonvanishing multipole moments.
Write down the potential at large distances as a finite expansion in Legendre polynomials.

(b) Determine the potential explicitly at any point in space, and show that near the origin, correct to $r^2$ inclusive,
\begin{eqnarray}
    \Phi(\va*{r}) \approx \frac{1}{4 \pi \epsilon_0} \Big[ \frac{1}{4} - \frac{r^2}{120} P_{2}(\cos{\theta}) \Big]
.\end{eqnarray}

}

\sol{

We can make a general multipole expansion in spherical harmonics:
\begin{eqnarray}
    \Phi(\va*{r}) = \frac{1}{\epsilon_0} \sum_{l=0}^{\infty} \sum_{m=-l}^{l} \frac{1}{2l+1} Y_{lm}(\theta,\phi) \int \dd[3]{\va*{r}'} \frac{r_{<}^{l}}{r_{>}^{l+1}} Y_{lm}^{*}(\theta',\phi') \rho(\va*{r}')
.\end{eqnarray}
For our specific charge density
\begin{eqnarray}
    \Phi(\va*{r}) = \frac{1}{64 \pi \epsilon_0} \sum_{l=0}^{\infty} \sum_{m=-l}^{l} \frac{1}{2l+1} Y_{lm}(\theta,\phi) \int_{0}^{\infty} \dd{r'} \frac{r_{<}^{l}}{r_{>}^{l+1}} r'^{4} e^{-r} \int \dd{\Omega'} Y_{lm}^{*}(\theta',\phi') \sin^2{\theta'}
.\end{eqnarray}
The angular integration can be done as follows:
\begin{eqnarray}
\begin{aligned}
    \int \dd{\Omega} Y_{lm}^{*}(\theta,\phi) \sin^2{\theta} &= \sqrt{\frac{2l+1}{4 \pi} \frac{(l-m)!}{(l+m)!}} \underbrace{ \int_{0}^{2\pi} \dd{\phi} e^{i m \phi} }_{ 2\pi \delta_{m_0} } \int_{0}^{\pi} \dd{\theta} \sin{\theta} P_{l}^{m}(\cos{\theta}) (1 - \cos^2{\theta}) \\
    &= \sqrt{\pi(2l+1)} \delta_{m 0} \int_{-1}^{1} \dd{x} P_{l}(x) \frac{2}{3}[ P_0(x) - P_2(x) ] \\
    &= \frac{4\sqrt{\pi}}{3} \Big[ \delta_{l 0} - \frac{1}{\sqrt{5}}\delta_{l 2} \Big] \delta_{m_0}
.\end{aligned}
\end{eqnarray}
Putting this into our multipole expansion, we find
\begin{eqnarray}
\begin{aligned}
    \Phi(\va*{r}) &= \frac{1}{64 \pi \epsilon_0} \frac{4 \sqrt{\pi}}{3} \Bigg[ Y_{00}(\theta,\phi) \Bigg( \int_{0}^{\infty} \dd{r'} \frac{r'^{4}}{r_{>}} e^{-r'} \Bigg) - \frac{1}{5\sqrt{5}} Y_{20}(\theta,\phi) \Bigg( \int_{0}^{\infty} \dd{r'} \frac{r_{<}^{2}}{r_{>}^{3}} r'^{4} e^{-r'} \Bigg) \Bigg] \\
    &= \frac{1}{96 \pi \epsilon_0} \Bigg[ P_{0}(\cos{\theta}) \Bigg( \int_{0}^{\infty} \dd{r'} \frac{r'^{4}}{r_{>}} e^{-r'} \Bigg) - \frac{1}{5} P_{2}(\cos{\theta}) \Bigg( \int_{0}^{\infty} \frac{r_{<}^2}{r_{>}^3} r'^{4} e^{-r'} \Bigg) \Bigg]
.\end{aligned}
\end{eqnarray}
To find the potential far away from the origin, we can take $r_{<} = r'$ and $r_{>} = r$ (valid for $r$ where $e^{-r}$ is essentially zero), giving
\begin{eqnarray}
\begin{aligned}
    \Phi(\va*{r}) &= \frac{1}{96 \pi \epsilon_0} \Bigg[ \frac{4!}{r} P_0(\cos{\theta}) - \frac{6!}{5r^3} P_{2}(\cos{\theta}) \Bigg] \\
    &= \frac{1}{4 \pi \epsilon_0 r} \Bigg[ 1 - \frac{6 P_{2}(\cos{\theta})}{r^2} \Bigg]
\end{aligned}
.\end{eqnarray}

(b) If we are close to the origin, then we must be a bit more careful and split the integration over two portions:
\begin{eqnarray}
    \int
.\end{eqnarray}

}

\prob{2}{

Calculate the multipole moments $q_{lm}$ of the charge distributions shown as parts (a) and (b).
Try to obtain results for the nonvanishing moments valid for all $l$, but in each case find the first \textit{two} sets of nonvanishing moments at the very least.

(c) For the charge distribution of the set (b) write down the multipole expansion for the potential.
Keeping only the lowest-order term in the expansion, plot the potential in the $xy$-plane as a function of distance from the origin for distances greater than $a$.

(d) Calculate directly fro Coulomb's law the exact potential for (b) in the $xy$-plane.
Plot it as a function of distance and compare with the result found in part (c).
Divide out the asymptotic form in parts (c) and (d) to see the behavior at large distances more clearly.

}



\end{document}
