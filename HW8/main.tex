\def\duedate{\today}
\def\HWnum{8}
\documentclass[10pt,a4paper]{book}

% custom section formatting
\usepackage{titlesec}
\titleformat{\chapter}[display]
{\normalfont\Large\filcenter\sffamily}
{\titlerule[1pt]%
\vspace{1pt}%
\titlerule
\vspace{1pc}%
\LARGE\MakeUppercase{\chaptertitlename} \thechapter}
{1pc}
{\titlerule
\vspace{1pc}%
\Huge}

% appendix handling
\usepackage[toc,page]{appendix}
    
% encoding for file and font
\usepackage[utf8]{inputenc}
\usepackage[T1]{fontenc}

% math formatting/tools
\usepackage{amsmath}
\usepackage{amssymb}
\usepackage{mathtools}
\usepackage[arrowdel]{physics}

% unit formatting
\usepackage{siunitx}
\AtBeginDocument{\RenewCommandCopy\qty\SI}

% figure formatting/tools
\usepackage{graphicx}
\usepackage{float}
\usepackage{subcaption}
\usepackage{multirow}
\usepackage{import}
\usepackage{pdfpages}
\usepackage{transparent}
\usepackage{currfile}

\NewDocumentCommand\incfig{O{1} m}{
    \def\svgwidth{#1\textwidth}
    \import{./Figures/\currfiledir}{#2.pdf_tex}
}

\newcommand{\bef}{\begin{figure}[h!tb]\centering}
\newcommand{\eef}{\end{figure}}

\newcommand{\bet}{\begin{table}[h!tb]\centering}
\newcommand{\eet}{\end{table}}

% hyperlink references 
\usepackage{hyperref}
\hypersetup{
    colorlinks=true,
    linkcolor=blue,
    filecolor=magenta,
    urlcolor=cyan,
    pdftitle={Physics 1 Notes},
    pdfauthor={Richard Whitehill},
    pdfpagemode=FullScreen
}
\urlstyle{same}

\newcommand{\eref}[1]{Eq.~(\ref{eq:#1})}
\newcommand{\erefs}[2]{Eqs.~(\ref{eq:#1})--(\ref{eq:#2})}

\newcommand{\fref}[1]{Fig.~(\ref{fig:#1})}
\newcommand{\frefs}[2]{Fig.~(\ref{fig:#1})--(\ref{fig:#2})}

\newcommand{\aref}[1]{Appendix~(\ref{app:#1})}
\newcommand{\sref}[1]{Section~(\ref{sec:#1})}
\newcommand{\srefs}[2]{Sections~(\ref{sec:#1})-(\ref{sec:#2})}

\newcommand{\tref}[1]{Table~(\ref{tab:#1})}
\newcommand{\trefs}[2]{Table~(\ref{tab:#1})--(\ref{tab:#2})}

% tcolorbox formatting/definitions
\usepackage[most]{tcolorbox}
\usepackage{xcolor}
\usepackage{xifthen}
\usepackage{parskip}

\definecolor{peach}{rgb}{1.0,0.8,0.64}

\DeclareTColorBox[auto counter, number within=chapter]{defbox}{O{}}{
    enhanced,
    boxrule=0pt,
    frame hidden,
    borderline west={4pt}{0pt}{green!50!black},
    colback=green!5,
    before upper=\textbf{Definition \thetcbcounter \ifthenelse{\isempty{#1}}{}{: #1} \\ },
    sharp corners
}

\newcommand*{\eqbox}{\tcboxmath[
    enhanced,
    colback=black!10!white,
    colframe=black,
    sharp corners,
    size=fbox,
    boxsep=8pt,
    boxrule=1pt
]}

\newtcolorbox[auto counter, number within=chapter]{exbox}{
    parbox=false,
    breakable,
    enhanced,
    sharp corners,
    boxrule=1pt,
    colback=white,
    colframe=black,
    before upper= \textbf{Example \thetcbcounter:}\,,
    before lower= \textbf{Solution:}\,,
    segmentation hidden
}

\newtcolorbox{resbox}{
    enhanced,
    colback=black!10!white,
    colframe=black,
    boxrule=1pt,
    boxsep=0pt,
    top=2pt,
    ams nodisplayskip,
    sharp corners
}


\begin{document}

\prob{1}{

A hollow right circular cylinder of radius $b$ has its axis coincident with the $z$-axis and its ends at $z = 0$ and $z = L$.
The cylindrical surface is made of two equal half-cylinders, one at potential $V$ and the other at potential $-V$, so that
\begin{eqnarray}
    V(\phi,z) = \begin{cases}
        V & -\pi/2 < \phi < \pi /2 \\
        -V & \pi/2 < \phi < 3 \pi /2
    .\end{cases}
\end{eqnarray}

(a) Find the potential inside the cylinder.

(b) Assuming $L \gg b$, consider the potential at $z = L/2$ as a function of $s$ and $\phi$ and compare it with two-dimensional Problem 1 from Pset 5.

}

\sol{

For this, we begin by looking at separable solutions of the form
\begin{eqnarray}
    \Phi(s,\phi,z) = R(s) T(\phi) Z(z)
.\end{eqnarray}
For the $z$-dependence, we have,
\begin{eqnarray}
    Z_{n}(z) = \sin{k_{n} z}
,\end{eqnarray}
where $k_{n} = \pi n / L$ such that the end faces of the cylinder are grounded.
The angular dependence is given by
\begin{eqnarray}
    T_{m}(\phi) = A_{m} \sin{m \phi} + B_{m} \cos{m \phi}
,\end{eqnarray}
and finally the radial solutions are from the modified Bessel equation
\begin{eqnarray}
    R_{nm}(s) = I_{m}(k_{n}s)
.\end{eqnarray}
Note the second solution $K_{m}$ is discarded since this leads to singular behavior at $s = 0$.

Thus, the series solution for the potential inside the cylinder is
\begin{eqnarray}
    \Phi(s,\phi,z) = \sum_{n=1}^{\infty} \frac{B_{0n}}{2} I_0(k_{n}s) \sin(k_{n}z) + \sum_{m=1}^{\infty} \sum_{n=1}^{\infty} I_{m}(k_{n}s) \sin(k_{n}z) [ A_{nm} \sin{m\phi} + B_{nm} \cos{m\phi} ]
.\end{eqnarray}
We then have a Fourier series for the angular and $z$-dependence of $\Phi$.
Applying the BC at $s = b$, we have
\begin{eqnarray}
    V(\phi,z) = \sum_{n=1}^{\infty} \frac{B_{0n}}{2} I_{0}(k_{n} b) \sin(k_{n}z) + \sum_{m=1}^{\infty} \sum_{n=1}^{\infty} I_{m}(k_{n} b) \sin(k_{n} z) [ A_{nm} \sin{m\phi} + B_{nm} \cos{m \phi} ]
.\end{eqnarray}
Exploiting the orthogonality of sines and cosines, we then have
\begin{align}
    A_{nm} &= \frac{2}{\pi L I_{m}(k_{n}b)} \int_{0}^{L} \dd{z} \int_{0}^{2 \pi} \dd{\phi} V(\phi,z) \sin(m \phi) \sin(\pi n z / L) \\
    B_{nm} &= \frac{2}{\pi L I_{m}(k_{n}b)} \int_{0}^{L} \dd{z} \int_{0}^{2 \pi} \dd{\phi} V(\phi,z) \cos(m \phi) \sin(\pi n z / L)
.\end{align}
Putting this into the potential we have 
\begin{align}
    \Phi(s,&\phi,z) = \frac{1}{\pi L} \sum_{n=1} \frac{I_{0}(n \pi s / L)}{I_0(n \pi b / L)} \sin(n \pi z / L) \Bigg[ \int_{0}^{L} \dd{z'} \int_{0}^{2\pi} \dd{\phi'} V(\phi',z') \sin(n\pi z' / L) \Bigg] \nonumber \\
                   &+ \frac{2}{\pi L} \sum_{m=1}^{\infty} \sum_{n=1}^{\infty} \frac{I_{m}(n \pi s / L)}{I_{m}(n \pi b / L)} \sin(n \pi z / L) \Bigg[ \int_{0}^{L} \dd{z'} \int_{0}^{2\pi} \dd{\phi'} V(\phi',z') \sin(n\pi z' / L) \cos[m(\phi - \phi')] \Bigg]
\end{align}
Observe that the first term ($m = 0$) is identically zero since the integral over $\phi'$ gives zero.
Our problem then reduces to evaluating the following integral:
\begin{eqnarray}
\begin{aligned}
    V &\int_{0}^{L} \sin(n\pi z' / L) \dd{z'} \Bigg[ \int_{-\pi/2}^{\pi/2} - \int_{\pi/2}^{3\pi/2} \Bigg] \dd{\phi'} \cos[m(\phi-\phi')] \\
      &= V \frac{L}{n\pi} [1 - (-1)^{n}] [1 - (-1)^{m}] \int_{-\pi/2}^{\pi/2} \cos[m(\phi-\phi')] \dd{\phi'} \\
      &= V \frac{L}{n \pi} [1 - (-1)^{n}] [1 - (-1)^{m}] \frac{1}{m} \Big\{ \sin[m(\pi/2 - \phi)] + \sin[m(\pi/2 + \phi)] \Big\} \\
      &= V \frac{2L}{n m \pi} [1 - (-1)^{n}] [1 - (-1)^{m}] \cos{m\phi} \sin{m\pi/2}
.\end{aligned}
\end{eqnarray}
From this expression, we can see that only odd terms in $n,m$ contribute, giving
\begin{eqnarray}
    \eqbox{ \Phi(s,\phi,z) = \frac{16}{\pi^2} V \sum_{n,m=0}^{\infty} \frac{(-1)^{m}}{(2n+1)(2m+1)} \frac{I_{m}\{(2n+1) \pi s / L\}}{I_{m}\{(2n+1) \pi b / L\}} \cos(m\phi) \sin\Bigg[\frac{(2n+1) \pi z}{L} \Bigg] }
.\end{eqnarray}

(b) If we have $z = L/2$, the potential becomes
\begin{eqnarray}
    \Phi(s,\phi) = \frac{16}{\pi^2} V \sum_{n,m=0}^{\infty} \frac{(-1)^{n+m}}{(2n+1)(2m+1)} \frac{I_{m}\{ (2n+1)\pi s/L \}}{I_{m}\{ (2n+1) \pi b/L \}} \cos(m\phi)
.\end{eqnarray}
Taking $L \gg b$, we can use $I_{\nu}(x) \rightarrow (x/2)^{\nu}/\Gamma(\nu+1)$ when $x \ll 1$.
Additionally, in this limit, $s \leq b$, so
\begin{eqnarray}
\begin{aligned}
    \Phi(s,\phi) &\approx \frac{16}{\pi^2} V \sum_{n,m=0}^{\infty} \frac{(-1)^{n+m}}{(2n+1)(2m+1)} \Big( \frac{s}{b} \Big)^{m} \cos(m\phi) \\
                 &= \frac{16}{\pi^2} V \underbrace{ \sum_{n=0}^{\infty} \frac{(-1)^{n}}{2n+1} }_{\pi/4} \Re\Bigg\{ \underbrace{ \sum_{m=0}^{\infty} \frac{(-1)^{m}}{2m+1} \Big( \frac{s}{b} e^{i\phi} \Big)^{m} }_{ \arctan( \frac{s}{b}e^{i\phi}) } \Bigg\} \\
                 &= \frac{4}{\pi} V \Re\Bigg\{ \arctan(\frac{s}{b} e^{i\phi}) \Bigg\}
.\end{aligned}
\end{eqnarray}

We now have to do just a touch of complex analysis.
Let $w = \arctan{z}$.
Then
\begin{gather}
    z = \tan{w} = \frac{\sin{w}}{\cos{w}} = \frac{1}{i}\frac{e^{iw} - e^{-iw}}{e^{iw} + e^{-iw}} = -i \frac{e^{2iw} - 1}{e^{2iw} + 1} \\
    iz = \frac{e^{2iw} - 1}{e^{2iw} + 1} \\
    e^{2iw} = \frac{1 + iz}{1 - iz} \\
    \Rightarrow w = \arctan{z} = \frac{1}{2i} \ln\Big[ \frac{1 + iz}{1 - iz} \Big]
.\end{gather}
Thus, we have
\begin{eqnarray}
    \Re(\arctan{z}) = \frac{1}{2} \arg\Bigg\{ \frac{1 + iz}{1 - iz} \Bigg\} = \frac{1}{2} \arctan(\frac{2\Re(z)}{1 - |z|^2})
.\end{eqnarray}

Finally, we can say that
\begin{eqnarray}
    \Phi(s,\phi) \approx \frac{2V}{\pi} \arctan( \frac{2s/b}{1 - (s/b)^2} \cos{\phi} ) = \frac{2V}{\pi} \arctan( \frac{2sb}{b^2 - s^2} \cos{\phi} )
,\end{eqnarray}
which is exactly the result of problem 1 from Pset 5 with $V_1 = V$ and $V_2 = -V$.


}


\prob{2}{

An infinite, thin, plane sheet of conducting material has a circular hole of radius $a$ cut in it.
A thin, flat disc of the same material and slightly smaller radius lies in the plane, filling the hole, but separated from the sheet by a very narrow insulating ring.
The disc is maintained at a fixed potential $V$, while the infinite sheet is kept at zero potential.

(a) Using appropriate cylindrical coordinates, find an integral expression involving Bessel functions for the potential at any point above the plane.

(b) Show that the potential a perpendicular distance $z$ above the center of the disc is
\begin{eqnarray}
    \Phi_{0}(z) = V \Bigg( 1 - \frac{z}{\sqrt{a^2 + z^2}} \Bigg)
.\end{eqnarray}

(c) \textbf{Extra credit}: Show that the potential a perpendicular distance $z$ above the edge of the disc is
\begin{eqnarray}
    \Phi_{a}(z) = \frac{V}{2} \Bigg[ 1 - \frac{kz}{\pi a} K(k) \Bigg]
,\end{eqnarray}
where $k = 2a/(z^2 + 4a^2)^{1/2}$, and $K(k)$ is the complete elliptic integral of the first kind.

}

\sol{

(a) Let us take the $z$-axis to be through the center of the disc.
The separable solutions are then of the form
\begin{eqnarray}
    \Phi_{m k}(s,\phi,z) = e^{-kz} [ A_{m}(k) \sin{m \phi} + B_{m}(k) \cos{m\phi} ] J_{m}(ks)
.\end{eqnarray}
Unlike in other problems, however, the spectrum of allowed $k$ values is continuous ($k > 0$), not discrete.
The values of $m$, though, is restricted to the positive integers and zero.
Putting this all together, we have a generic solution for $z > 0$ of the form
\begin{align}
    \Phi(s,\phi,z) = \int_{0}^{\infty} \dd{k} e^{-kz} \frac{B_0(k)}{2} J_0(ks) + \sum_{m=1}^{\infty} \int_{0}^{\infty} \dd{k} e^{-kz} [ A_{m}(k) \sin{m \phi} + B_{m}(k) \cos{m \phi} ] J_{m}(ks)
.\end{align}
The coefficients are determined via the Hankel transforms in $s$ and Fourier series in $\phi$ as
\begin{align}
    A_{m}(k) &= \frac{k}{\pi} \int_{0}^{\infty} \dd{s} s \int_{0}^{2\pi} \dd{\phi} V(s,\phi) \sin{m \phi} J_{m}(ks) \\
    B_{m}(k) &= \frac{k}{\pi} \int_{0}^{\infty} \dd{s} s \int_{0}^{2\pi} \dd{\phi} V(s,\phi) \cos{m \phi} J_{m}(ks)
.\end{align}
The potential can then be written
\begin{eqnarray}
\begin{aligned}
    \Phi(s,\phi,z) = \frac{V}{\pi} \int_{0}^{a} \dd{s'} s' &\int_{0}^{2 \pi} \dd{\phi'} \int_{0}^{\infty} \dd{k} k e^{-kz} \\
                   &\times \Bigg\{ \frac{1}{2} J_0(ks')J_0(ks) + \sum_{m=1}^{\infty} \cos[m(\phi - \phi')] J_{m}(ks') J_{m}(ks) \Bigg\}
.\end{aligned}
\end{eqnarray}
Since there is no $\phi$ dependence in the potential, we can simplify this a bit further as
\begin{eqnarray}
    \Phi(s,\phi,z) = V \int_{0}^{a} \dd{s'} s' \int_{0}^{\infty} \dd{k} k e^{-kz} J_0(ks') J_0(ks)
.\end{eqnarray}
Additionally, recall that $x J_0(x) = (x J_1(x))'$ and $J_1(0) = 0$, so
\begin{eqnarray}
\begin{aligned}
    \Phi(s,\phi,z) &= V \int_{0}^{\infty} \dd{k} e^{-kz} J_0(ks) \int_{0}^{a} \dd{s'} ks'J_0(ks') \\
                   &= V a \int_{0}^{\infty} \dd{k} e^{-kz} J_0(ks) J_1(ka)
.\end{aligned}
\end{eqnarray}
At this point -- as far as I can tell -- we cannot simplify this any further in full generality.

(b) If we take $s = 0$ and restrict our attention to the $z$-axis
\begin{eqnarray}
\begin{aligned}
    \Phi(z) &= Va \int_{0}^{\infty} \dd{k} e^{-kz} J_1(ka) = Va \int_{0}^{\infty} \dd{k} e^{-kz} \frac{ka}{2} \sum_{j=0}^{\infty} \frac{(-1)^{j}}{j! (j+1)!} \frac{(ka)^{j}}{4^{j}} \\
            &= \frac{V}{2} \sum_{j=0}^{\infty} 
\end{aligned}
.\end{eqnarray}

}


\prob{3}{

A hollow right cylinder of radius $R$ has its axis coincident with the $z$-axis and its ends at $z = 0$ and $z = L$.
The potential on the end faces is zero, while the potential on the cylindrical surface is specified by
\begin{eqnarray}
    \Phi(s,\phi,z)|_{s = R} = V \sin{\phi} \sin(\pi z / L)
.\end{eqnarray}
Find the potential $\Phi(s,\phi,z)$ inside the cylinder.

}







\end{document}
